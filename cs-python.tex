\section{Python}
\subsection{Scoping}
There're four levels:
\begin{itemize}
\item current scope
\item parent scope
\item module scope (global)
\item built-in scope
\end{itemize}

\texttt{nonlocal} keyword specify this variable should be referenced
to the parent scope.  But, this will not reach global.  Instead, the
=global= keyword declares the listed variables to be in the module
level scope.

\begin{quote}
The nonlocal statement causes the listed identifiers to refer to previously bound variables in the nearest enclosing scope excluding globals.
\end{quote}

As an example:
\begin{lstlisting}[language=python]
var = 0 # global

def outer():
  var = 1 # parent
  def inner():
    nonlocal var
    var = 2 # local
    global var
    var =3
  inner()
  # var = 2

outer()
# global var = 3
\end{lstlisting}



\subsection{Collection}

\subsubsection{String}

\paragraph{Concatenation}
\begin{itemize}
\item concatenate two strings directly by =+=.
\item need to convert integer to string before concatenate: =s + str(35)=
\end{itemize}

\paragraph{split}
\begin{description}
\item [str.split(sep=None)] default by white space
\item [str.strip()] strip out white space at both begin and end
\item [str.replace(old, new)] replace /all/.
\item [str.startswith(s)]
\item [str.endswith(s)]
\end{description}

\subsubsection{tuple}
TODO
\subsubsection{ List}
\paragraph{Slicing}
The slicing syntax is \texttt{l[start:end:step]}.
The slicing will return a \textit{new} list. Change to that list will not change the original one.

\begin{lstlisting}
l[4]
l[4:]
l[::2]
l[:-1]
\end{lstlisting}

However, assign to the slicing itself /will change/ the original one:
\begin{lstlisting}
l[1:2] = [4,5,6]
\end{lstlisting}

Also, assign to a new variable only assign the reference:
\begin{lstlisting}
a = [1,2,3]
b = a # only a reference
\end{lstlisting}

\paragraph{create a list}
\begin{description}
\item [range(stop)]
\item [range(start, stop[, step])]
\end{description}


\subsubsection{Dictionary}
Create:
\begin{lstlisting}
x = {'a': 1, 'b': 2}
\end{lstlisting}

Dictionary is not sorted. Use \texttt{collections.OrderedDict} if you
want this feature.  Basically it remember the order when the elements
are inserted.

\begin{lstlisting}
import collections
od = collections.OrderedDict(sorted(d.items()))
\end{lstlisting}

Merge two dictionary (=x= and =y=):
\begin{lstlisting}
z = x.copy()
z.update(y)
\end{lstlisting}


\subsubsection{Set}
\begin{lstlisting}
s = set()
s.add(x)
if x in s:
  pass
\end{lstlisting}




\subsection{Algorithm}
\subsubsection{sort}
TODO

sort a dictionary by value:
\begin{lstlisting}
sorted(dict1, key=dict1.get) # => list
sorted(dict1, key=dict1.get, reverse=True)
\end{lstlisting}


\subsection{Function}
\subsubsection{variadic parameter}
use =*args= syntax, and =args= will be a /tuple/:
\begin{lstlisting}
  def foo(*args):
    for a in args:
      print a
\end{lstlisting}

use =**args= to capture all /keyword arguments/.

\begin{lstlisting}
def bar(**kwargs):
  for a in kwargs:
    print a, kwargs[a]
\end{lstlisting}

Combine them together:
\begin{lstlisting}
def foobar(kind, *args, **kwargs):
  pass
\end{lstlisting}

Also, there's a concept for the reverse thing: unpack argument list from a list, with =*list=:
\begin{lstlisting}
def foo(a,b):
  pass

l = [1,2]
foo(*l)
\end{lstlisting}

on python3, this syntax can appear on left side
\begin{lstlisting}
first, *rest = [1,2,3,4]
first,*l,last = [1,2,3,4]
\end{lstlisting}

\subsection{Exception}
To give a quick feel:
\begin{lstlisting}
try:
  pass
except TypeError as e: # capture the exception into a variable
  pass
except AnotherError: # does not capture
  pass
except: # all exception
  pass
else: # if doesn't raise an exception
  pass
finally:
  pass
\end{lstlisting}

\subsection{Lambda}
\begin{lstlisting}
lambda x : x+2
lambda x: x%2==0
\end{lstlisting}

The usage of lambda is often in /map/ and /filter/.
\begin{description}
\item [map(lambdaexp, mylist)] will execute the lambda expression on
  each element of the list, and return a list containing the results.
\end{description}

\subsection{Packaging}
Exposing API: the following only expose =foo= but not =bar=.
\begin{lstlisting}
__all__ = ['foo']
def foo():
  pass
def bar():
  pass
\end{lstlisting}

\subsubsection{importing}
The local structure directory must contain the \verb$__init__.py$ file to be able to import.
\begin{lstlisting}
|-- main.py
|-- mypackage
    |-- __init__.py
    |-- a.py
    |-- b.py
    |-- subdir
        |-- __init__.py
        |-- c.py
\end{lstlisting}

The import statements should be:
\begin{lstlisting}
from mypackage import a
from mypackage.b import foo as myfoo
from mypackage.subdir import c
\end{lstlisting}

\subsection{Thread}
\begin{lstlisting}
from threading import Thread

class MyThread(Thread):
  def __init__(self, arg):
    Thread.__init__(self)
    self.arg = arg
  def run(self):
    pass

t = MyThread(arg)
t.start()
\end{lstlisting}





\subsection{Type}
\begin{description}
\item [not True]
\item [i += 1]
\end{description}
\subsubsection{conversion}
\begin{description}
\item [string to integer] \texttt{int('45')}
\item [integer to string] \texttt{str(45)}
\item [ASCII to char] \texttt{chr(100)} returns \texttt{'d'}
\item [char to ASCII] \texttt{ord('d')} returns \texttt{100}
\end{description}

\subsection{Black Tech}
If else or:
\begin{lstlisting}
var = d.get('key') or 0
# is equal to:
var = d.get('key') if d.get('key') else 0
\end{lstlisting}

list comprehension

\begin{lstlisting}
even_squares = [x**2 for x in l if x%2 == 0]
\end{lstlisting}

\subsection{Pep8}
Indent:
\begin{itemize}
\item *function and class* should be separated by *2 lines*
\item *In a class*, function should be separated by *1 line*
\item 1 space before and after variable assignment
\end{itemize}

Naming
\begin{itemize}
\item function, variable, attribute: \verb$func_var_attr$
\item protected instance attributes: \verb$_protected_field$
\item private instance attributes: \verb$__private_field$
\item class and exception: \verb$ClassExceptionName$
\item module level constants: \verb$CONSTANT$
\item instance method of class should use =self= as first parameter, refer to the object
\item class method should use =cls= as first parameter, refer to the class
\end{itemize}

Expression

\begin{itemize}
\item use \texttt{a is not b} instead of \sout{\texttt{not a is b}}
\item use \texttt{if not list} instead of \sout{\texttt{if len(list) == 0}}
\end{itemize}

Import
\begin{itemize}
\item always use absolute path
\item if must use relative, use =from . import foo= instead of +=import foo=+
\end{itemize}

\subsubsection{document}
One can use one line or multi-line document.
The doc string can be retrieved by \verb$func.__doc__$.
\begin{lstlisting}
def func():
  """one line doc"""

def func():
  """The outline

  The above empty line is required.
  Here's the detailed documentation.
  """
\end{lstlisting}

\subsection{IO}
\begin{lstlisting}
print('xxx', end='')
\end{lstlisting}

\begin{lstlisting}
  f = open('text.txt')
  f.read() # return all content

  f = open('text.txt')
  for line in f:
      print(line)

  with open('a.txt') as f:
      for line in f:
          print(line)
\end{lstlisting}

read from stdin:
\begin{lstlisting}
for line in sys.stdin:
  print(line)
\end{lstlisting}

get command line argument: \texttt{sys.argv}






\subsection{Operating System}

\subsubsection{Work filesystem}
\begin{lstlisting}
import os
for root,dirs,files in os.walk('.'):
  for f in files:
    print f
\end{lstlisting}

\begin{description}
\item [os.path.abspath('relative/path/to/file')]
\item [os.path.exists("/path/to/file")]
\item [os.rename('old', 'new')]
\end{description}


\subsubsection{Shell command}
\begin{description}
\item [os.system] simply run command
\item [os.popen] access to input output
  \begin{lstlisting}
    stream = os.popen("some command")
    stream.read()
  \end{lstlisting}
\item [subprocess.Popen]
  \begin{lstlisting}
    p = subprocess.Popen("echo Hello World", shell=True, stdout=subprocess.PIPE)
    p.stdout.read()
    s = subprocess.check_output('wc -l', stdin=p.stdout)
  \end{lstlisting}
\item [subprocess.call] this is the same as =subprocess.Popen= except that it waits and gives return code.
  \begin{lstlisting}
    return_code = subprocess.call("echo Hello World", shell=True, stdout=subprocess.DEVNULL)
  \end{lstlisting}
\end{description}

\subsection{Third party libraries}
\subsubsection{argparse}
\begin{lstlisting}
import argparse
parser = argparse.ArgumentParser()
parser.add_argument('-q', '--query', help='query github api', require=True)
parser.add_argument('-d', '--download', help='do download', action='store_true')
args = parser.parse_args()
\end{lstlisting}

\subsubsection{urllib}
\begin{lstlisting}
from urllib import request
import json

url = 'https://api.github.com'
api = '/search/repositories'
query = 'language:C&stars:>10&per_page='+size
response = request.urlopen(url+api+"?q="+query)

s = response.read().decode('utf8')
j = json.loads(s)
# j will be a mix of list and dict
\end{lstlisting}

\subsubsection{XML}
\begin{lstlisting}
import xml.etree.ElementTree as ET
root = ET.fromstring(s)
# XPath
nodes = root.findall('{http://www.sdml.info/srcML/src}function')
for node in nodes:
  # do with node
  pass
\end{lstlisting}

APIs
\begin{description}
\item [node.find(XPath)]
\item [node.findall(XPath)]
\item [node.get(Attribute)]
\item [node.text]
\end{description}





%%% Local Variables:
%%% TeX-master: "cheatsheet"
%%% End:
