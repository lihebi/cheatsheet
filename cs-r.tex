\subsection{R}



\subsection{Tmp}
\begin{description}
\item [str] show the structure of arbitrary type
\item [na.omit] can omit the NA values in data frame
\end{description}

A data frame is a list of equal-length vectors.  When getting the data
from read.csv, the result is a data frame.  Use =names= to work on
data frames will emit the names.
\begin{itemize}
\item Since it is a list, using =[]= to index will give also the list,
  a.k.a. data frame, retaining names.  You can use a vector as index.
\item Using =[[]]= to index will give the value, dropping names.  You
  cannot use a vector as index.
\end{itemize}

\subsection{Setting up}

\subsubsection{Packages}
\begin{description}
\item [library()] see the list of installed packages.
\item [library(class)] load the package =class=.
\item [search()] see the list of loaded packages.
\item [install.packages()] and =update.packages()= install and update packages.
\end{description}

\subsubsection{Emacs ESS}
To start ESS session, run command =S=.  ESS will create a command
interface as a buffer.  Execute =?foo= will open the =R-doc= for the
function =foo=.

There's a babel in org mode for R, so just =C-c C-c= would work.  This
will prompt to create a session.  One special for this babel is you
can evaluate by line, using =C-<RET>= in the edit buffer.

   Keep the session using the header:
\begin{lstlisting}
#+PROPERTY: session *R*
\end{lstlisting}

   To export a graph:

\begin{lstlisting}
:file image.png :results output graphics :exports both
\end{lstlisting}

   To export ordinary result:

\begin{lstlisting}
:exports both
\end{lstlisting}

   To export some summary data:

\begin{lstlisting}
:exports both :results output org
\end{lstlisting}

\subsubsection{Interactive command}
=setwd("/absolute/path")= set the working directory.

In the interactive command line interface,
\begin{description}
\item [q()] to exit, 
\item [help(lm)] to show help message for a built-in function.
\item [?lm] shows documentation
\item [??solve] shows documentation
\item [example(topic)] shows the examples.
\item [help.start()] opens the html documentation page.
\end{description}


IO:
\begin{description}
\item [source("a.R")] loads a script.
\item [sink("a.lis")] redirects the output to a file, and =sink()= restore that to standard output.
\end{description}


Objects:
\begin{description}
\item [ls()] list the objects currently stored.
\item [objects()] create and store current objects.
\item [rm(x, y, z, ink, junk, temp, foo, bar)] remove objects
\end{description}


\subsection{Objects}
Objects have mode and length.  The =typeof= gets the type of an
object, while =mode= retrieves the mode of an object.  =length= gets
the length of the vector.

Objects have attributes.
\begin{description}
\item [names] is used for indexing
\item [dim] is the dimension of a matrix
\item [dimnames] is the character names of the dimensions
\end{description}

The missing values are =NA=, tested by =is.na=.  Illegal computations
produces =NaN=, e.g. =0/0=.

Compound objects are /factors/ (a.k.a. enumerators) and /data
frames/ (lists of objects all have the same length).
\begin{description}
\item [Vectors] are the array of objects of the same mode.
\item [List] can contain the objects of different modes.
\item [data frame] is a list with class "data.frame".
\end{description}

\subsubsection{Vector}
Create a vector by =c=: =c(10.4, 5.6, 3.1)=.
This connects elements /end to end/, e.g. =c(x, 0, x)=.
By a sequence =1:30=: the same as: =c(1, 2, ..., 29, 30)=.
Colon operator can also specify a backward sequence: =30:1=
Colon operator has higher priority: ~2*1:15~ is the same as =c(2, 4, …, 28, 30)=
The colon operator works, if you want a more flexible and powerful
expression, =seq= is what you want:
\begin{description}
\item [seq(2,10)] produces \texttt{2:10}.
\item [seq(-5, 5, by=.2)]
\item [seq(length=51, from=-5, by=.2)]
\end{description}

Repeating can be done by =rep=:
\begin{description}
\item [x <- c(1,2,3)]
\item [rep(x, times=5)] \verb$[1,2,3,1,2,3,...]$
\item [rep(x, each=5)] \verb$[1,1,1,1,1,2,2,2,2,2,...]$
\end{description}
\subsubsection{Indexing}
Inside =[]=, it can be a number or character, or a vector of them.
For vectors, =[]= returns the element.  For lists, =[]= will return
the the element inside a list, while =[[]]= will return the single
element.  =[]= does not allow a vector as index.

If the index is integer, will select based on the position, start from
1.  If it is negative, it means the elements other than those index.
The index 0 will return empty.  Other numeric values will be converted
to integer towards zero.

   If the index is logical vector, the true ones will be returned.
   If the index is character, it is compared, /partially/, with the /names/ attributes of the vector.
   \texttt{\$} can be used for indexing with character.
   The empty index =[]= will returns the entire vector with /irrelevant/ attributes removed.
   The only retained ones are the =names=, =dim= and =dimnames= attributes.
\begin{lstlisting}
fruit <- c(5, 10, 1, 20)
names(fruit) <- c("orange", "banana", "apple", "peach")
lunch <- fruit[c("apple","orange")]
# matrix
dim(z) <- c(3,5,100)~
z[2,,]
z[,,]
\end{lstlisting}

\subsubsection{data example}

\begin{lstlisting}
  ## (HEBI: Command line arguments)
  args = commandArgs(trailingOnly=TRUE)
  csvfile = args[1]
  csv = read.csv(csvfile, header=TRUE)

  total_test <- dim(csv)[[1]]
  sub = subset(csv, reach_code>=5)
  total_reach_poi <- dim(sub)[[1]]
  sub = subset(csv, reach_code==5 & status_code == 1)
  total_fail_poi <- dim(sub)[[1]]

  sub <- sub[1:(length(csv)-2)]
  ## (HEBI: callin ga function)
  funcs = TransferFunction(sub);

  ## (HEBI: define a function)
  Constant <- function(data) {
      ## (HEBI: return value as a vector)
      ret <- c()
      i <- 1
      ## (HEBI: a for loop using the vector as range)
      for (i in c(1:length(data))) {
          col = data[i];
          ## (HEBI: Get the name of a column)
          name = names(col);
          if (substr(name, 1, 6) == "output") {
              ## (HEBI: remove of NA)
              newcol = col[!is.na(col)];
              if (length(newcol) > 2) {
                  value <- newcol[1]
                  ## (HEBI: check the value of the vector is all the same)
                  if (length(newcol[newcol != value]) == 0) {
                      ## (HEBI: pushing a new value to the return vector)
                      ret <- c(ret, paste("name = ",  value))}}}}
      return(ret)}
\end{lstlisting}

\subsection{Operators}
\begin{description}
\item [arithmetic] \texttt{+-*/}, \verb$^$ for exp, \verb$%%$ for modulus
\item [matrix] \verb$%*%$ matrix product, \verb$%o%$ outer product
\item [logic] \verb$!$, \verb$&, |$ for vector, \verb$&&, ||$ for no vector
\item [relative] \verb$>, <, ==, <=, >=$
\item [general] \verb$<-, ->$ assignments, \texttt{\$} list subset,
  \verb$:$ sequence, \verb$~$ for model formula
\end{description}


Built-in functions:
\begin{description}
\item =log=, =exp=, =sin=, =cos=, =tan=, =sqrt=
\item =min=, =max=
\item =range=: same as =c(min(x),max(x))=
\item =length(x)=, =sum(x)=, =prod(x)= (product)
\item =mean(x)=: =sum(x)/length(x)=
\item [var(x)] \texttt{sum((x-mean(x))\^2)/(length(x)-1)}
\item ~sort(x)~: increasing order
\item ~order()~ or ~sort.list()~
\item =paste()= function takes an arbitrary number of arguments and
  concatenates them one by one into character strings.
\end{description}


\subsection{Control Structure}
The compound statements are the same as C, can be a single statement without the braces.
\subsubsection{Conditional}
\begin{description}
\item [if] =if (STMT) STMT else if (STMT) STMT else STMT=
\item [Switch] =switch (STMT, LIST)=
  \begin{itemize}
  \item the STMT is first evaluated
  \item if the value is within 1 and the length of the LIST, evaluate LIST[i], and return
  \item return NULL
  \item Notice that the LIST can be a comma separated argument of
    switch ... which means switch actually accepts =...=
\end{itemize}
\end{description}

\subsubsection{Loop}
\begin{description}
\item [for] =for (NAME in VECTOR) STMT=
\item [while] =while (STMT) STMT
\item [repeat] repeat STMT
\item [break, next]
\end{description}

\subsection{Evaluation rules}
\begin{description}
\item [recycling rules] the shortest list is recycled to the length of longest.
\item [dimensional attributes] the dimension of matrix must match. No recycle for a matrix.
\end{description}

\subsection{Function}
=function (ARGLIST) BODY=

The argument list can be a symbol, a ~symbol=value~, or a =...=.  The
body is a compound expression, surrounded with ={}=.  Function can be
assigned to a symbol.

The matching of formals and actual are pretty tricky.
\begin{enumerate}
\item exact matching on tags
\item partial matching on tags
\item positional matching for =...=
\end{enumerate}
Partial matching result must be unique, but the exact matched ones are
excluded before this step is entered.

\subsection{Quote}
The quote will wrap the expression into an object without evaluating it.
The resulting object has the mode of =call=.
The =eval= is used to evaluate it.
\subsection{Debugging}
To enter the debugger, a call to =browser= function suffices.
This allows you to browse the values at that point.
A more powerful debugger is by a call to =debug= with the function name as argument.
Each time that function is called, you enter the debug and can control the execution.
Tracing can be registered by =trace= or =untrace= with the name of the function.
It might need to be quoted in some case, so you'd better quote it, with double quotes.
Every time the function is invoked, the return value will be printed as trace.
\subsection{Data IO}
\begin{description}
\item [write]
\item [write.table]
\item [write.csv]
\item [read.table("filename", header=TRUE, sep=",")]
  \begin{itemize}
  \item this ignores blank lines,
  \item and expect the header to be one field less than the body.
  \item \verb$#$ as comments
  \end{itemize}
\end{description}

\subsection{Models}
  Linear model.

\begin{lstlisting}
fm = lm(y ~ x1 + x2, data = mydataframe)
\end{lstlisting}

The fitted model in the variable =fm= can be accessed by:
\begin{description}
\item [coef] extract the coefficients
\item [deviance] the Residual Sum of Square
\item [formula] extract the model formula
\item [plot] produce four plots: residuals, fitted values, diagnostics.
\item [predict(OBJECT, newdata=DATA.FRAME)] use the model to predict
\item [residuals] extract the residuals
\item [summary] print the summary
\end{description}

The models can be updated, if the formula only changes a little bit.
In the following example, the =.= means the corresponding part of the
original formula.
\begin{lstlisting}
fs <- lm(y~x1 + x2, data=mydata)
fs <- update(fs, . ~ . + x3)
fs <- update(fs, sqrt(.) ~ .)
\end{lstlisting}


\subsection{Packages}
\subsubsection{ggplot2}
\begin{lstlisting}
qplot(totbill, tip, geom="point", data=tips) # scatter plot
qplot(totbill, tip, geom="point", data=tips) + geom_smooth(method="lm") # with linear relationship line
qplot(tip, geom="histogram", data=tip) # histogram
qplot(tip, geom="histogram", binwidth=1, data=tips) # with custom binwidth
# box plots
qplot(sex, tipperc, geom="boxplot", data=tips)
qplot(smoker, tipperc, geom="boxplot", data=tips)
qplot(sex:smoker, tipperc, geom="boxplot", data=tips) # combine! plot the two sets of graph in two one graph
qplot(totbill, tip, geom="point", colour=day, data=tips) # scatter plot with colors, in regard to "day" column
\end{lstlisting}

\subsubsection{plot(x, y, ...)}
Possible =...= arguments:
\begin{description}
\item [type] what type of plot:
  \begin{description}
  \item [p] for points,
  \item [l] for lines,
  \item [b] for both,
  \item [h] for =histogram= like (or =high-density=) vertical lines,
  \end{description}
\item [main] an overall title for the plot: see =title=.
\item [xlab] a title for the x axis: see =title=.
\item [ylab] a title for the y axis: see =title=.
\end{description}

\subsubsection{dplyr}
A Grammar of Data Manipulation
\begin{itemize}
\item https://cran.r-project.org/web/packages/dplyr/index.html
\item https://cran.rstudio.com/web/packages/dplyr/vignettes/introduction.html
\end{itemize}





%%% Local Variables:
%%% TeX-master: "cheatsheet"
%%% End:
