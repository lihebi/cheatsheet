\section{\LaTeX}

\subsection{Installation}
To see what is your tex home: \texttt{kpsewhich -var-value=TEXMFHOME}
It should be something like "~/texmf".  Putting class and style file
into correct path inside that folder will enable global usage of the
class.  check whether it works or not: \texttt{kpsewhich
  sig-alternate-05-2015.cls} Typically you don't need to update
database, but if you want, Command to update the =ls-R= database:
\texttt{texhash} or \texttt{mktexlsr}

\begin{tabular}{@{}ll|ll@{}}
  \verb$\alpha$ & $\alpha$ & \verb$\theta$ & $\theta$ \\
  \verb$\phi$ & $\phi$ & \verb$\varphi$ & $\varphi$ \\
  \verb$\xi$ & $\xi$ & \verb$\mu$& $\mu$\\
  \verb$\pi$ & $\pi$ & \verb$\rho$ & $\rho$\\
  \verb$\sigma$ & $\sigma$ & \verb$\epsilon$ & $\epsilon$\\
  \verb$\partial$ & $\partial$\\
  \hline
  \verb$\quad$ & $\alpha\quad\beta$ & \verb$\qquad$ & $\alpha\qquad\beta$\\
  \hline
  \verb$\cup$ & $\cup$ & \verb$\bigcup$ & $\bigcup$ \\
  \verb$\cap$ & $\cap$ & \verb$\vee$ & $\vee$ \\
  \verb$\wedge$ & $\wedge$ & \verb$\in$ & $\in$ \\
  \verb$\notin$ & $\notin$ & \verb$\neg$ & $\neg$ \\
  \verb$\subset$ & $\subset$ & \verb$\subseteq$ & $\subseteq$\\
  \verb$\supset$ & $\supset$ & \verb$\supseteq$ & $\supseteq$ \\
  \verb$\le$ & $\le$ & \verb$\ge$ & $\ge$\\
  \verb$\neq$ & $\neq$ & \verb$\forall$ & $\forall$\\
  \verb$\exists$ & $\exists$\\
  \hline
  \verb$\leftarrow$ & $\leftarrow$ & \verb$\rightarrow$ & $\rightarrow$\\
  \verb$\Rightarrow$ & $\Rightarrow$ & \verb$\Leftarrow$ & $\Leftarrow$\\
  \verb$\Leftrightarrow$ & $\Leftrightarrow$ & \verb$\longrightarrow$ & $\longrightarrow$\\
  \hline
  \verb$\hat{a}$ & $\hat{a}$ & \verb$\vec{x}$ & $\vec{x}$\\
  \hline
  \verb$\infty$ & $\infty$ & \verb$\propto$ & $\propto$\\
  \verb$\lfloor$ & $\lfloor$ & \verb$\rfloor$ & $\rfloor$\\
  \verb$\lceil$ & $\lceil$ & \verb$\rceil$ & $\rceil$\\
  \verb$\sum$ & $\sum$ & \verb$\int$ & $\int$\\
  \verb$\ldots$ & $ \ldots$ & \verb$\frac{a}{b}$ & $\frac{a}{b}$\\
  \verb$\sqrt{n}$ & $\sqrt{n}$ & \verb$\overline{abc}$ & $\overline{abc}$\\
  \verb$\prod$ & $\prod$ & &  \\
  \hline
  \verb$\checkmark$ & $\checkmark$ & \verb$\times$ & $\times$\\
  \hline
  \verb$\tiny$ & \tiny tiny & \verb$\scriptsize$ & \scriptsize scriptsize\\
  \verb$\footnotesize$ & \footnotesize footnotesize& \verb$\small$ & \small small\\
  \verb$\normalsize$ & \normalsize normalsize& \verb$\large$ & \large large\\
  \verb$\Large$ & \Large Large& \verb$\LARGE$ & \LARGE LARGE\\
  \verb$\huge$ & \huge huge& \verb$\Huge$ & \Huge Huge
\end{tabular}



\subsubsection{General Syntax}

\begin{tabular}{@{}l|l|l@{}}
  \begin{lstlisting}
    \begin{enumerate}
    \item xxx
    \end{enumerate}
  \end{lstlisting} &
\begin{lstlisting}
  \begin{itemize}
  \item like this,
  \end{itemize}
\end{lstlisting}&
\begin{lstlisting}
  \begin{description}
  \item[Word] Definition
  \end{description}
\end{lstlisting}\\
\end{tabular}

\begin{tabular}{@{}l|l@{}}
\begin{lstlisting}
  \begin{table}
    \centering
    \begin{tabular}{l|r}
      Item & Quantity \\\hline
      Widgets & 42 \\
      Gadgets & 13
    \end{tabular}
    \caption{\label{tab:widgets}
      An example table.}
  \end{table}
\end{lstlisting} &
\begin{lstlisting}
  \begin{figure}
    \centering
    \includegraphics[width=0.3\textwidth]{frog.jpg}
    \caption{\label{fig:frog}caption.}
  \end{figure}
\end{lstlisting}
\end{tabular}

\begin{tabular}{@{}l|l@{}}
  \begin{lstlisting}
    \label{xxx}
    \ref{xxx}
    \label{xx:yy}
    \ref{xx:yy}
  \end{lstlisting}&
\begin{lstlisting}
\todo{comment in the margin!}
\todo[inline, color=green!40]{inline comment.}
\end{lstlisting}
\end{tabular}


\subsection{Package: listings}
Predefined languages:
Ada, Algol, Ant, Assembler, Awk, bash, Basic, C, C++, CIL, csh,
Delphi, Eiffel, erlang, Fortran, Gnuplot, Haskell, HTML, Java, Lisp,
LLVM, Lua, make, Matlab, Mercury, ML, Octave, Pascal, Perl, PHP,
PostScript, Prolog, Python, R, Ruby, S, Scala, sh, SQL, tcl, TeX,
VBScript, Verilog, VHDL, XML, XSLT


To make the code listing more pretty, the font needs to be changed.
\verb$\usepackage{courier}$.

global setting:

\begin{lstlisting}
\lstset{basicstyle=\ttfamily,breaklines=true}
% \lstset{frame=b}
\lstset{float,floatplacement=H,captionpos=b}
% \lstset{numbers=left}
\lstset{language=C}
\lstset{showstringspaces=false}
% \lstset{framextopmargin=10pt}
% \lstset{framextopmargin=50pt,frame=t}
% \lstset{float=htb,language=C,frame=single, basicstyle=\small, stringstyle=\ttfamily}
\lstset{escapeinside={(*@}{@*)}}
\end{lstlisting}


%%% Local Variables:
%%% TeX-master: "cheatsheet"
%%% End:
