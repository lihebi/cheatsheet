\section{TikZ}

Use of tikz packages.

\begin{lstlisting}
\usepackage{tikz}
\usetikzlibrary{shapes.multipart}
\end{lstlisting}

You can generate picture as pdf file.  To generate figure, add the
following lines, and execute: \texttt{pdflatex -shell-escape
  helium.tex}

\begin{lstlisting}
\usetikzlibrary{external}
\tikzexternalize
\tikzset{external/force remake}
\end{lstlisting}

There's also some library not supported. They need luatex.
\begin{lstlisting}
\usetikzlibrary{graphs,graphdrawing}
\usegdlibrary{layered}
\end{lstlisting}

To create a tikz picture, use
\begin{lstlisting}
  \begin{figure*}[ht]
    \centering
    \begin{tikzpicture}[options]
    \end{tikzpicture}
    \caption{\label{}}
  \end{figure*}
\end{lstlisting}

You can also create picture using \verb$\tikz$ command. It is the
same as begin and end tikzpicture, and put command inside. If only one
command, the curly braces are not required.
\begin{lstlisting}
\tikz[⟨options⟩]{⟨path commands⟩}
\end{lstlisting}

scope can be useful to specify styles.
\begin{lstlisting}
\begin{scope}[color=red]
\end{scope}
\end{lstlisting}

The most common errors for tikz are: miss semicolon; miss curly
braces; miss include tikz library.

Set options by \verb$\tikzset$.
\begin{lstlisting}
\tikzset{>=latex}
\tikzset{grid/.style={gray,very thin,opacity=1}}
\tikzset{anchor/.append code=\let\tikz@auto@anchor\relax}
\tikzset{label anchor/.style={tikz@label@post/.append
    style={anchor=#1}}}
\end{lstlisting}

\subsection{Coordinate}
\begin{lstlisting}
([shift={(2,-0.5)}] iflen.east)
\end{lstlisting}

Inside a path, the first one must be a coordinate. The following ones
can use \texttt{+} or \texttt{++}, meaning shift from current, as well
as shift and move current.

The general syntax is
\begin{lstlisting}
  (<coordinate system> cs:<list of key=value>)
\end{lstlisting}

The following coordinate systems are available:
\begin{description}
\item [canvas]
  \begin{description}
  \item [x]
  \item [y]
  \end{description}
\item [xyz]
  \begin{description}
  \item [x]
  \item [y]
  \item [z]
  \end{description}
\item [canvas polar]
  \begin{description}
  \item [angle]
  \item [radius]
  \item [x radius]
  \item [y radius]
  \end{description}
\item [xyz polar] same as above
  % \begin{description}
  % \item [angle]
  % \item [radius]
  % \item [x radius]
  % \item [y radius]
  % \end{description}
\item [xy polar] alias for \texttt{xyz polar}
\item [node]
  \begin{description}
  \item [name]
  \item [anchor] north, etc
  \item [angle] degreee, e.g. 90, -70
  \end{description}
\item [tangent]
  \begin{description}
  \item [node]
  \item [point]
  \item [solution]
  \end{description}
\end{description}

Using \texttt{calc} library, you can compute the coordinate.
\begin{lstlisting}
([⟨options⟩]$⟨coordinate computation⟩$)
($(a)+3*(1cm,0)$)
\end{lstlisting}



\subsection{Style}
style can be defined.
\begin{lstlisting}
  \begin{tikzpicture}[help lines/.style={blue!50,very thin}]
    \draw[help lines] (2,0) grid +(2,2);
  \end{tikzpicture}
\end{lstlisting}

\subsection{Path}
Add \texttt{-- cycle} at the end of a path will complete the path by
connecting back to the start.

\texttt{/tikz/every path} can be used to customize all path.

Operations
\begin{description}
\item [move-to] just coordinates
\item [line-to]
  \begin{description}
  \item [straight line] \texttt{-- (1,1)}
  \item [horizontal and vertical] \texttt{|-} or \texttt{-|}
  \end{description}
\item [curve-to] \texttt{.. controls <c> (and <d>) ..}
\item [rectangle] \texttt{(0,0) rectangle (1,1)}
\item [rounded corners]
\item [sharp corners]
\item [circle] \texttt{circle[⟨options⟩]}. It has option
  \texttt{/tikz/every circle}
  \begin{description}
  \item [x radius]
  \item [y radius]
  \item [radius]
  \item [at]
  \end{description}
\item [ellipse] ellipse[⟨options⟩] Same as circle
\item [arc]
  \begin{description}
  \item [start angle]
  \item [end angle]
  \item [delta angle]
  \end{description}
\item [grid] \verb$\draw[grid] (-2,0) grid (2,12);$
  \begin{description}
  \item [step]
  \item [xstep]
  \item [ystep]
  \item [help lines]
  \end{description}
\item [to] TODO
\item [foreach] TODO
\item [let] TODO
\end{description}

\subsection{Action}
\verb$\draw$ equals \verb$\path{draw}$
Action
\begin{description}
\item [draw]
\item [fill]
\item [filldraw]
\item [pattern] options: \texttt{pattern color}
  \begin{description}
  \item [dots]
  \item [fivepointed stars]
  \item [bricks]
  \end{description}
\item [shade] It has an option \texttt{shading
    angle=<degrees>}. Another option \texttt{shading=<name>} can be:
  \begin{description}
  \item [axis]
  \item [radial]
  \item [ball]
  \end{description}
\item [shadedraw]
\item [clip]
\item [useasboundingbox]
\end{description}


Graphic parameters
\begin{description}
\item [color] used for fill, draw, text. Can use \texttt{red!20!black}
  thanks to \texttt{color} package.
  \begin{description}
  \item [draw=<color>]
  \item [fill=<color>]
  \item [text=<color>] apply on text
  \end{description}

\item [draw] 
\item [line width] Line can be: ultra thin, very thin, thin,
  semithick, thick, very thick, ultra thick. Of course the more
  flexible way is using \texttt{line width = <dimension>}.
\item [dash patterns]
  \begin{description}
  \item [dash pattern=]
  \item [dash phase=]
  \item [solid]
  \item [<densely> <loosely> dotted]
  \item [<densely> <loosely> dashed]
  \item [<densely> <loosely> dash dot]
  \item [<densely> <loosely> dash dot dot]
  \end{description}
\item [double] with option \texttt{double distance=<dimension>}
\end{description}

\subsection{Arrows}
To add the arrow tips, first add \texttt{[->]} option for the tikz
environment.  The library \texttt{arrows.meta} defines etras arrow
types.


\subsection{Node \& Edge}

\begin{lstlisting}
node [⟨options⟩] (⟨name⟩) at (⟨coordinate⟩) {⟨contents⟩};
\end{lstlisting}

\verb$\path node$ equals \verb$\node$. \texttt{every node} and
\texttt{every <shape> node} is available.

Multi part node:
\begin{lstlisting}
  \node[circle split,draw,double,fill=red!20] {top \nodepart{lower} bot};
\end{lstlisting}

\subsubsection{Option}
\begin{description}
\item [align=left] have to have this to make the \verb$\\$ able to create
  newline
\item [shape=<shape name>] \texttt{shape=} is optional. Can be
  rectangle, circle.
\item [inner sep=<dimension>]
\item [inner xsep]
\item [inner ysep]
\item [outer sep]
\item [outer xsep]
\item [outer ysep]
\item [minimum height]
\item [minimum width]
\item [minimum size]
\end{description}


\subsubsection{Position}
\begin{description}
\item [anchor]
\item [above=<offset]
\item [below]
\item [left]
\item [right]
\item [above left]
\item [above right]
\item [below left]
\item [below right]
\item [centered]
\item [fit] \texttt{fit=(a) (b) (c)}
\end{description}

Placing node on edge
\begin{description}
\item [pos=<fraction>]
\item [auto]
\item [swap]
\item ['] alias of swap
\item [sloped]
\item [allow upside down]
\item [midway]
\item [near start] 0.25
\item [near end] 0.75
\item [very near start] 0.125
\item [very near end] 0.875
\item [at start]
\item [at end]
\end{description}

\subsubsection{Label \& Pin}
General syntax is \texttt{label=[<options>]<angle>:<text>}.
\texttt{every label/.style} is available.

Options:
\begin{description}
\item [label position=<angle>]
\item [absolute=boolean]
\item [label distance]
\end{description}

General syntax for pin is \texttt{pin=[<options>]<angle>:<text>}

options:
\begin{description}
\item [pin distance]
\item [every pin]
\item [pin position=<angle>]
\item [every pin edge]
\end{description}

The \texttt{quotes} package provides quote syntax:
\texttt{"<text>"<options>}. The option \texttt{every label quotes} is
available.
\begin{lstlisting}
\tikz \node ["my label" red, draw] {my node};
\end{lstlisting}


\subsection{Graph}
This requires the \texttt{graphs} library.

\begin{description}
\item [graph syntax] \verb$\path graph[⟨options⟩]⟨group specification⟩$
\item [group spec] \verb${[⟨options⟩]⟨list of chain specifications⟩}$
  The nodes can be grouped by surround them with curly braces. The
  entry and exit points are computed for the chains in and out the
  group.

\item [chain spec] consists of list of node spec seperated by edge
  spec
\item [edge spec] each accept an optional \texttt{[<options>]} after them.
  \begin{description}
  \item [->] directed edge
  \item [--] undirected edge
  \item [<-] backward
  \item [<->] bidirected
  \item [-!-] no edge
  \end{description}
\item [node spec] \verb$"<node name>"/"<text>"[<options>]$ The nodes
  used have a special notation. The quotes are required only when the
  node name contains special symbols. If the node name is empty, it
  will be given a random name.
\end{description}

TODO: GROW DOWN

\subsection{Tree}

\begin{lstlisting}
\begin{tikzpicture}
  \node {root}
    child {node {left}}
    child {node {right}
      child {node {child}}
      child {node {child}}
    };
\end{tikzpicture}
\end{lstlisting}

\subsection{Package}
\subsubsection{shapes.multipart}
\begin{lstlisting}
mynode/.style={split, rectangle split parts=2}
\end{lstlisting}
\subsubsection{patterns}
\subsubsection{positioning,fit,calc}
\subsubsection{decorations.pathmorphing}
\subsubsection{decorations.pathreplacing}
\subsubsection{quotes}
\subsubsection{graphs}
\subsubsection{arrows.meta}




%%% Local Variables:
%%% TeX-master: "cheatsheet"
%%% End:
