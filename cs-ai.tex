\section{AI}

\subsection{Agent Concept}

I'll give some questions in the homework ..

\begin{quote}
  An agent that senses only partial information about the state cannot
  be perfectly rational.
\end{quote}

False. The vacuum-cleaning agent is perfectly rational, but it senses
only partial information, i.e. it doesn't observe the state of the
square that is adjacent to it.

\begin{quote}
  There exist task environment in which no pure reflex agent can
  behave rationally.
\end{quote}
True. Purely reflective behavior does not take the percept history
into account, only the most recent percepts.

\begin{quote}
  There exists a task environment in which every agent is rational.
\end{quote}
True.  A task environment in which there are no decisions to be made:
all actions will receive the same output.

\begin{quote}
  The input to an agent program is the same as the input to the agent
  function
\end{quote}
False. The input to agent program: current percept while the input to
agent function: entire precept history.

\begin{quote}
  Every agent function is implementable by some program/machine
  combination
\end{quote}
False. As the textbook said: "No machine can tell /in general/ whether
a given program will return an answer on a given input or run
forever."

\begin{quote}
  suppose an agent selects its action uniformly at random from the set
  of possible actions.  There exists a deterministic task environment
  in which this agent is rational.
\end{quote}
True.  In the environment that all actions will produce same output,
it is rational.  Actually all agents are rational in such environment.

\begin{quote}
It is possible for a given agent to be perfectly rational in two distinct task environments.
\end{quote}

True.  There's recently a kickstarter project that produces dice with
more than 6 faces.  If an agent is rational in a N face dice bet game,
it will perform equally well in a 6-face dice or a 8-face dice.
\begin{quote}
Every agent is rational in an un-observable environment.
\end{quote}
False.  A vacuum agent that can move will be rational, but the one
that does not move is not.

\begin{quote}
A perfectly rational poker-playing agent never loses.
\end{quote}
False.  Two such perfectly rational agent play against each other will
give one lose.
      
\subsubsection{Agent function v.s. Agent program}
\begin{quote}
Can there be more than one agent program that implements a given agent
function? Give an example, or show why one is not possible.
\end{quote}
There are infinite agent programs that implement a given agent
function.  If an agent function acts only depend on previous $n$
percepts.  Than, the agent implementations that have n or more memory
will always produce the same action.

\begin{quote}
Are there agent functions that cannot be implemented by any agent program?
\end{quote}

Yes. As the textbook said: "No machine can tell /in general/ whether a
given program will return an answer on a given input or run forever."

\begin{quote}
  Given a fixed machine architecture, does each agent program
  implement exactly one agent function?
\end{quote}

Yes. A program implements a mapping from percepts to actions.  The
same percept will only result in one action.

\begin{quote}
  Given an architecture with n bits of storage, how many different
  possible agent programs are there?
\end{quote}

There would be $a^{2^n}$ possible programs; $2^n$ possible states and
$a$ choices for each state.

\begin{quote}
  Suppose we keep the a gent program fixed but speed up the machine by
  a factor of two. Does that change the agent function?
\end{quote}

No. The speed does not have influence on the produced action.






\subsection{State changing and implementation}
\subsubsection{hanoon jugs}
\begin{quote}
  Give a complete problem formulation for each of the following.
  Choose a formulation that is precise enough to be implemented.
  d. You have three jugs, measuring 12 gallons, 8 gallons, and 3 gallons, and a water faucet.
  You can fill the jugs up or empty them out from one to another or onto the ground.
  You need to measure out exactly one gallon.
\end{quote}

Define a 3-tuple =(x,y,z)= where x,y,z are the amount of water in the
three jugs.
\begin{itemize}
\item Initial state: =(0,0,0)=.
\item Action:
  \begin{itemize}
  \item FILL: given values =(x,y,z)= , generate =(12,y,z)=, =(x,8,z)=,
    =(x,y,4)=
  \item EMPTY: given values =(x,y,z)= , generate =(0,y,z)=, =(x,0,z)=,
    =(x,y,0)=
  \item POUR: Given value =(x,y)=, let ~t = min(x+y, cap(y))~, pour x
    into y, generate: =(x-(t-y), t)=
  \end{itemize}
\item Cost function: Number of actions.
\end{itemize}

\subsubsection{野人与传教士}
三个野人,三个传教士,一艘船。如何过河?


\paragraph{Formulate}
\begin{quote}
  Formulate the problem precisely. making only those distinctions
  necessary to ensure a valid solution. Draw a diagram of the complete
  state space.
\end{quote}

state
\begin{itemize}
\item =(M1,C1,B1,M2,C2,B2)= where =M1,C1,B1= is the number of missionaries, cannibals, boats on the left side,
  =(M2,C2,B2)= is the corresponding number on the right side.
\item The start state is =(3,3,1,0,0,0)=.
\item The goal: =(0,0,0,3,3,1)=
\end{itemize}
Action: =(m,c,b)= on left side: where b means the change of boat, m
and c means the change of missionaries and cannibals.  The action
allows the boat number B1 or B2 to change from 1 to 0, along with M
and C on the side move to the other side by one or two.
     
\begin{lstlisting}
  (-1 0 -1)
  (0 -1 -1)
  (-2 0 -1)
  (0 -2 -1)
  (-1 -1 -1)

  (1 0 1)
  (0 1 1)
  (2 0 1)
  (0 2 1)
  (1 1 1)
\end{lstlisting}
The complete state space.

In the table below, the striped items are those that cannibals will
eat missionaries.  The state that is not reachable
(e.g. =(3,3,0,0,0,1)=) is not striped out.

\begin{lstlisting}
| =(3 3 1 0 0 0)= | +=(3 2 1 0 1 0)=+ | +=(3 1 1 0 2 0)=+ | +=(3 0 1 0 3 0)=+  |
| +=(2 3 1 1 0 0)=+ | =(2 2 1 1 1 0)= | +=(2 1 1 1 2 0)=+ | +=(2 0 1 1 3 0)=+  |
| +=(1 3 1 2 0 0)=+ | +=(1 2 1 2 1 0)=+ | =(1 1 1 2 2 0)= | +=(1 0 1 2 3 0)=+  |
| +=(0 3 1 3 0 0)=+ | +=(0 2 1 3 1 0)=+ | +=(0 1 1 3 2 0)=+ | =(0 0 1 3 3 0)=  |
| =(3 3 0 0 0 1)= | +=(3 2 0 0 1 1)=+ | +=(3 1 0 0 2 1)=+ | +=(3 0 0 0 3 1)=+  |
| +=(2 3 0 1 0 1)=+ | =(2 2 0 1 1 1)= | +=(2 1 0 1 2 1)=+ | +=(2 0 0 1 3 1)=+  |
| +=(1 3 0 2 0 1)=+ | +=(1 2 0 2 1 1)=+ | =(1 1 0 2 2 1)= | +=(1 0 0 2 3 1)=+  |
| +=(0 3 0 3 0 1)=+ | +=(0 2 0 3 1 1)=+ | +=(0 1 0 3 2 1)=+ | =(0 0 0 3 3 1)=  |
\end{lstlisting}

\paragraph{Solve}
\begin{quote}
Implement and solve the problem optimally using an appropriate search algorithm.
Is it a good idea to check for repeated states?
\end{quote}

The solution:
\begin{lstlisting}
  (3,3,1,0,0,0)
  -> (2,2,0,1,1,1)
  -> (3,2,1,0,1,0)
  -> (3,0,0,0,3,1)
  -> (3,1,1,0,2,0)
  -> (1,1,0,2,2,1)
  -> (2,2,1,1,1,0)
  -> (0,2,0,3,1,1)
  -> (0,3,1,3,0,0)
  -> (0,1,0,3,2,1)
  -> (0,2,1,3,1,0)
  -> (0,0,0,3,1,0)
\end{lstlisting}
Yes, we should check repeated states to avoid infinite recursion.

\paragraph{discussion}
\begin{quote}
  Why do you think people have a hard time solving this puzzle,
  given that the state space is so simple?
\end{quote}

\begin{enumerate}
\item It is hard to manually work it out.
\item the repeat states need to be removed, which increase difficulty for manual solving.
\end{enumerate}



\subsection{Search Algorithm}

\subsubsection{branching factor, BFS, DFS}
\begin{quote}
  3.26 Consider the unbounded version of regular 2D grid shown in Figure 39.
  The start state is at the origin, (0,0), and the goal state is at (x,y).
\end{quote}

\begin{itemize}
\item What is the branching factor b in the state space?
  \begin{itemize}
  \item Since it is a 2D grid, there're 4 directions for each node. The branching factor is 4.
  \end{itemize}
\item How many distict states are there at depth k (for k > 0)?
  \begin{itemize}
  \item For depth 1, there're 1+4 states;
  \item For depth 2, there're 1+4+8 states;
  \item For depth 3, there're 1+4+8+12 states;
  \item For depth k, there're $1 + 4 + 8 + .. + 4k = 2k^2 + 2k + 1$
  \end{itemize}
\item What is the maximum number of nodes expanded by breadth-first
  tree search?
  \begin{itemize}
  \item The depth of the goal is |x|+|y|, and if we allow the loopy
    states on the search tree, we will have 4 branches for each
    node. Thus the maximum total nodes to be expanded:
    $1 + 4^1 + 4^2 + .. + 4^(|x|+|y|)$
  \end{itemize}
\item What is the maximum number of nodes expanded by breadth-first
  graph search?
  \begin{itemize}
  \item For graph search, we only expand nodes that are not in
    exploded set.  The expanded nodes will be the distinct state of
    depth |x|+|y|: 1 + 4 + 8 + .. + 4k
  \end{itemize}
\item Is h= |u-x| + |v-y| an admissible heuristic for a state at
  (u,v)? explain.
  \begin{itemize}
  \item Yes. Because it never overestimates the cost: it is the
    optimal path from (u,v) to (x,y) in a 2D grid given that all links
    cost 1.
  \end{itemize}
\item How many nodes are expanded by A* graph search using h?
  \begin{itemize}
  \item It is |x|*|y|. Because all the paths in the rectangle are
    optimal paths.
  \end{itemize}
\item Does h remain admissible if some links are removed?
  \begin{itemize}
  \item Yes. Removing links can only make the best path longer if
    possible, so h remains an underestimate.
  \end{itemize}
\item Does h remain admissible if some links are added between
  nonadjacent states?
  \begin{itemize}
  \item No. We could add some links that makes the optimal solution
    shorter.  Thus h would overestimate the cost.
  \end{itemize}
\end{itemize}


\subsection{Formulation}
\subsubsection{Floor planning}
\begin{quote}
6.4 Given the precise formulations for each of the following as a Constraint Satisfaction Problems:
Rectilinear floor-planning: find non-overlapping places in a large rectangle for a number of smaller rectangles.
\end{quote}

\begin{description}
\item [Variables]
  \begin{itemize}
  \item WIDTH and HEIGHT for the large rectangle.
  \item $R_i$ for each rectangles, $R_{i}$.w and $R_{i}.h$ for the width and height of $R_i$ respectively.
  \item the position $P_i$ for \textit{top-left} corner of the rectangle $R_i (P_{i}.x$ and $P_{i}.y$ for the co-ordinates)
  \end{itemize}
\item [Domains] ${P_{i}.x \in [0, WIDTH]}$
\item [Constraints] the four corners of $R_i$ should not be inside the
  area of $R_j$, for each $i \neq j$
  \begin{itemize}
  \item  for each $i \neq j$
  \item  for each corner $(x,y)$ in
    $\{(P_{i}.x, P_{i}.y)$, $(P_i.x + R_i.w, P_i.y)$, $(P_i.x, P_i.y + R_i.h)$, $(P_i.x+R_i.w, P_i.y + R_i.h)\}$
  \item  $\neg (P_{j}.x < x < P_{j}.x + R_{j}.w \cap P_{j}.y < y < P_{j}.y + R_{j}.h)$
  \end{itemize}
\end{description}

\subsubsection{class scheduling}
\begin{quote}
  Class scheduling: There is a fixed number of professors and
  classrooms, a list of classes to be offered, and a list of possible
  time slots for classes.  Each professor has a set of classes that he
  or she can teach.
\end{quote}

\begin{description}
\item [Variables]
  \begin{itemize}
  \item $P_i$ for each professor, with $P_i.classes$ be the set of classes he or she can teach
  \item $R_i$ for each room
  \item $C_i$ for each class
  \item $T_i$ for each time slot
  \item Assignment $A_i$, the i-th assignment, is a 4-tuple $(A_i.prof, A_i.room, A_i.class, A_i.time)$.
  \end{itemize}
\item [Domain]
  \begin{itemize}
  \item $A_i.prof \in {P_j}$
  \item $A_i.room \in {R_j}$
  \item $A_i.class \in {C_j}$
  \item $A_i.time \in {T_j}$
  \end{itemize}
\item [Constraint]
  \begin{itemize}
  \item $A_i.class \in A_i.prof.classes$ for all i
  \item $\neg (A_i.time = A_j.time \cap A_i.prof = A_j.prof)$ for all $i \neq j$
  \item $\neg (A_i.time = A_j.time \cap A_i.room = A_j.room)$ for all $i \neq j$
  \end{itemize}
\end{description}


\subsubsection{linving in 5 houses}
\begin{quote}
  Consider the following logic puzzle: In five houses, each with a
  different color, live five persons of different nationalities, each
  of whom prefers a different brand of candy, a different drink, and a
  different pet.  Given the following facts, the questions to answer
  are "where does the zebra live, and in which house do they drink
  water?"  Discuss different representations of this problem as a CSP.
  Why would one prefer one representation over another?
  \begin{itemize}
  \item The Englishman lives in the red house.
  \item The Spaniard owns the dog.
  \item The Norwegian lives in the first house on the left.
  \item The green house is immediately to the right of the ivory house.
  \item The man who eats Hershey bars lives in the house next to the man with the fox.
  \item Kit Kats are eaten in the yellow house.
  \item The Norwegian lives next to the blue house.
  \item The Smarties eater owns snails.
  \item The Snickers eater drinks orange juice.
  \item The Ukrainian drinks tea.
  \item The Japanese eats Milky Ways.
  \item Kit Kats are eaten in a house next to the house where the horse is kept.
  \item Coffee is drunk in the green house.
  \item Milk is drunk in the middle house.
  \end{itemize}
\end{quote}


\paragraph{First representation}
This representation is based on the house.
\begin{description}
\item [Variables and domains] H for the houses
  \begin{description}
  \item [.n] nationality
  \item [.b] hold brand of candy
  \item [.lb] the one lived in this house like the brand of candy
  \item [.c] color
  \item [.d] hold drink
  \item [.ld] the man lived in this house like the drink
  \item [.pet] hold pet
  \item [.index] index of the house from left to right
  \end{description}
\item [Domains]
  \begin{itemize}
  \item h.n $\in$ Englishman, Spaniard, etc.
  \item h.c $\in$ red, green, etc.
  \item h.b $\in$ ivory, smarties, etc.
  \item h.d $\in$ water, tea, etc.
  \item h.p $\in$ dog, fox, snail, etc.
  \item h.index $\in$ [1,5]
\end{itemize}
\item [Constraints]
  \begin{itemize}
  \item The Englishman lives in the red house. \\
    $\neg h.c=red \cup h.n=Englishman$
  \item The Spaniard owns the dog. \\
    $\neg h.n=Spanisard \cup h.pet=dog$
  \item The Norwegian lives in the first house on the left.\\
    $\neg h.n = Norwegian \cup h.index = 1$
  \item The green house is immediately to the right of the ivory house.\\
    $\neg (h1.c = green \cap h2.b = ivory) \cup h1.index = h2.index + 1$
  \item The man who eats Hershey bars lives in the house next to the man with the fox.\\
    $\neg (h1.lb=Hersheybar \cap \cap h2.pet = fox) \cup | h1.index-h2.index | = 1$
  \item Kit Kats are eaten in the yellow house.\\
    $\neg h.c=yellow \cup h.lb = KitKats$
  \item The Norwegian lives next to the blue house.\\
    $\neg (h1.n=Norwegian \cap h2.c = blue) \cup | h1.index - h2.index |=1$
  \item The Smarties eater owns snails.\\
    $\neg h.lb=smarties \cup h.p = snails$
  \item The Snickers eater drinks orange juice.\\
    $\neg h.lb=snickers \cup h.ld = OrangeJuice$
  \item The Ukrainian drinks tea.
    $\neg h.n=Ukrainian \cup h.ld=tea$
  \item The Japanese eats Milky Ways.
    $\neg h.n=Japanese \cup h.lb=MilkyWays$
  \item Kit Kats are eaten in a house next to the house where the horse is kept.
    $\neg (h1.b=Kats \cap h2.pet=horse) \cup |h1.index-h2.index|=1$
  \item Coffee is drunk in the green house.
    $\neg h.d=coffee \cup h.c=green$
  \item Milk is drunk in the middle house.
    $\neg h.d=milk \cup h.index=3$
  \end{itemize}
\end{description}

\paragraph{Second Representation}
This representation is based on the Person.  The domains and
constraints are similar.

\begin{description}
\item [Variables] P for person
  \begin{itemize}
  \item .n : nationality
  \item .b : in his house, the candy that holds
  \item .lb : the brand of candy he likes
  \item .c : the color of his house
  \item .d: the drink held in his house
  \item .ld: the drink he likes
  \item .pet: his pet
  \item .index: the index of his house
  \end{itemize}
  
\end{description}

    
\paragraph{Comparison}
Similarly, we can also derive the representation based on other
variables, e.g. drink, candy, etc.  One would prefer one to the other
based on the query type he want.  For example, if the query is based
on the house, e.g. which house hold some drink, he would prefer the
house-based representation.  Similarly if the query is based on
person, e.g. what pet does the Englishman keeps, he would prefer the
person-based representation.  But essentially they are the same.



\subsection{Propositional Logic}
\subsubsection{Simple}
\begin{quote}
  Given the following, can you prove that the unicorn is mythical?
  How about magical? Horned?

  If the unicorn is mythical, then it is immortal,
  but if it is not mythical, then it is a mortal mammal.
  If the unicorn is either immortal or a mammal, then it is horned.
  The unicorn is magical if it is horned.
\end{quote}

The formula writes:
\begin{itemize}
\item $\neg mythical \vee immortal$
\item $mythical \vee mammal$
\item $\neg (immortal \vee mammal) \vee horned$
\item $\neg horned \vee magical$
\end{itemize}

We have the enumerated truth table:
\begin{lstlisting}
| mythical | immortal | mammal | horned | magical |
|----------+----------+--------+--------+---------|
| T        | T        | T      | T      | T       |
| T        | T        | F      | T      | T       |
| F        | T        | T      | T      | T       |
| F        | F        | T      | T      | T       |
\end{lstlisting}

Based on these, we can derive:
\begin{itemize}
\item We cannot prove unicorn is mythical.
\item We can prove it is magical.
\item We can prove it is horned.
\end{itemize}

\subsubsection{Party resolution}
\begin{quote}
  7.18 Consider the following sentence:
  $$
  ((Food \Rightarrow Party) \vee (Drinks \Rightarrow Party))
  \Rightarrow ((food \wedge drinks) \Rightarrow Party)
  $$
  \begin{enumerate}
  \item Determine, use enumeration, whether the sentence is
    valid, satisfiable (but neg valid), or unsatisfiable.
  \item Convert the left-hand and right-hand sides of the main
    implication into CNF, Showing each step, and explain how the
    results confirm your answer to (a)
  \item Prove your answer to (a) using resolution.
  \end{enumerate}
\end{quote}

\paragraph{1}
It is Valid.
The enumeration:

\begin{tabular}{l|l|l|l|l|l}
  F & D & P & left hand & right hand & left $\Rightarrow$ right\\
  \hline
  T & T & T & T         & T          & T\\
  T & F & T & T         & T          & T\\
  F & T & T & T         & T          & T\\
  F & F & T & T         & T          & T\\
  \hline
  T & T & F & F         & F          & T\\
  T & F & F & T         & T          & T\\
  F & T & F & T         & T          & T\\
  F & F & F & T         & T          & T
\end{tabular}

\paragraph{2}
left-hand CNF:
\begin{eqnarray}
  ((F \Rightarrow P) \vee (D \Rightarrow P)\\
  \spadesuit (\neg F \vee P) \vee (\neg D \vee P)\\
  \spadesuit (\neg F \vee P \vee \neg D \vee P)\\
  \spadesuit (\neg F \vee \neg D \vee P)\\
\end{eqnarray}

right-hand CNF:
\begin{eqnarray}
  (F \wedge D) \Rightarrow P\\
  \spadesuit \neg (F \wedge D) \vee P\\
  \spadesuit (\neg F \vee \neg D) \vee P\\
  \spadesuit \neg F \vee \neg D \vee P
\end{eqnarray}

As we can see, they are exactly the same. Thus the production is valid.

\paragraph{3}
We can prove it by proving the negation is unsatisfiable.

$$\neg ((F \Rightarrow P) \vee (D \Rightarrow P) \Rightarrow ((F \vee D) \Rightarrow P))$$

\begin{eqnarray}
  \spadesuit \neg (((F \Rightarrow P) \vee (D \Rightarrow P)) \Rightarrow ((F \wedge D) \Rightarrow P))\\
  \spadesuit \neg ( \neg ((F \Rightarrow P) \vee (D \Rightarrow P)) \vee ((F \wedge D) \Rightarrow P))\\
  \spadesuit ((F \Rightarrow P) \vee (D \Rightarrow P)) \wedge (\neg ((F \wedge D) \Rightarrow P))\\
  \spadesuit ((\neg F \vee P) \vee (\neg D \vee P)) \wedge (\neg (\neg (F \wedge D) \vee P))\\
  \spadesuit (\neg F \vee \neg D \vee P) \wedge (F \wedge D) \wedge \neg P\\
  \spadesuit (\neg F \vee \neg D \vee P) \wedge F \wedge D \wedge \neg P
\end{eqnarray}
This resolves to empty clause, thus the original sentence is valid.



\subsection{Adversarial Search}

This is multiple agents, also known as /game/.

\subsubsection{Minimax Algorithm}
There're two players, Min and Max, each takes turn to execute.  Max
moves first.

\subsubsection{The optimal strategies}

% \begin{equation*}
%   MINIMAX-VALUE(n) = \left\{
%   \begin{array}{r1}
%     Utility(n) & \text {if n is terminal},\\
%     max_{s \in succ(n)} MINIMAX-VALUE(s) & \text{if n is a max node},\\
%     min_{s \in succ(n)} MINIMAX-VALUE(s) & \text{if n is a min node}.
%   \end{array} \right .
% \end{equation*}

Basically it recursively solve the problem.  The Utility function is
the payoff.  It actually list the tree of state space, and it is
optimal.

\subsubsection{Alpha-Beta pruning}
The problem of minimax algorithm is its node grow exponential.  This
algorithm is used to prune the subtree that does not affect the
result.

This is similar for MiniMax algorithm
\begin{itemize}
\item $\alpha$ is the value of the best choise so far, for max, init
  from $-\infty$
\item $\beta$ is the best value for min, init from $+\infty$
\end{itemize}

There're two procedures:
\begin{description}
\item [Alpha-Beta-Search(state)] returns an action. state is the current state.
\item [Max-Value(state, $\alpha$, $\beta$)] returns a utility value
\item [Min-Value(state, $\alpha$, $\beta$)] returns a utility value
\end{description}

\begin{lstlisting}
Alpha-Beta-Search(state) {
  v = Max-Value(state, -999, +999);
  return action in ACTIONS with value v;
}
Max-Value(state, alpha, beta) {
  v = INT_MIN;
  for each a in ACTIONS(state) do
    v = Max(v, Min-Value(result(s,a), alpha, beta));
    if v >= beta then return v;
    alpha = MAX(alpha,v);
  return v;
}
\end{lstlisting}

\subsection{Constraint Satisfaction Problems}
It seems to formulate the search problems in a uniformed
representation:
\begin{description}
\item [X] a set of variables
\item [D] each has a domain of values
\item [C] a set of constraints for each of the variable
\end{description}

The goal is to find the assignment of values to the variables, that satisfies the constraints.

\subsubsection{Advantage}
it uses /general purpose heuristic/ rather than /problem-specific/ ones.

\subsubsection{Variations}
\begin{itemize}
\item continuous or discrete domain
\item finite or infinite domain
\item linear or non-linear constraint
\item unary or binary or high order constraint
\end{itemize}




\subsection{Logic}
\subsubsection[entailment]
Entailment: $\beta \models \alpha$, reads: the sentence $\beta$ entails
the sentence $\alpha$ if and only if $\alpha$ is true in all worlds where
$\beta$ is true.

\subsubsection{Propositional logic}
a.k.a. boolean logic.

logical equivalence;

\begin{tabular}{l|l}
  a                                   & b\\
  \hline
  $\alpha \wedge \beta$                 & $\beta \wedge \alpha$\\
  $(\alpha \wedge \beta) \wedge \gamma$ & $\alpha \wedge (\beta \wedge \gamma)$\\
  $\alpha \Rightarrow \beta$            & $\neg \beta \Rightarrow \neg \alpha$\\
  $\alpha \Rightarrow \beta$            & $\neg \alpha \vee \beta$\\
  $\alpha \Leftrightarrow \beta$        & $(\alpha \Rightarrow \beta) \wedge (\beta \Rightarrow \alpha)$
\end{tabular}

\begin{itemize}
\item A sentence is /valid/ if it is true in all models.
  Deduction theorem: $KB \models \alpha$ iff $KB \Rightarrow
  \alpha$ is valid
\item A sentence is satisfiable if it is true in /some/ models.
  $KB \models \alpha$ iff $KB \wedge \neg \alpha$ is unsatisfiable.
\end{itemize}

Proof method:
\begin{itemize}
\item inference rules: transform the sentences to a normal form
\item model checking: truth table
\end{itemize}

A clause is a disjunction of literals.
Factoring: the result clause keeps only one copy of each literal.

Conjunction: $\wedge$
Disjunction: $\vee$
CNF: conjunctive normal form. Conjunction of (disjunctions of literals).


Resolution algorithm: proof by contradiction.
I.e. to show $KB \models \alpha$, we show $KB \wedge \neg \alpha$ is unsatisfiable.
The naming resolution is because, the pair of complementary literals is resolved.

Definite clause: disjunction of literals with exactly one positive literal.


\subsubsection{First Order Logic}
$\wedge$ is the natural connective with $\exists$.
Using $\Rightarrow$ as the main connective with $\exists$ often causes errors:

$$\exists x At(x,ISU) \Rightarrow Smart(x)$$

is true if there's anyone who is not at ISU, which may not be what you want.

Properties:
\begin{lstlisting}
  | a                   | b                   | result    |
  |---------------------+---------------------+-----------|
  | \forall x \forall y | \forall y \forall x |           |
  | \exists x \exists y | \exists y \exists x |           |
  | \exists x \forall y | \forall y \exists x | not equal |
  | \forall x           | \neg \exists x \neg |           |
  | \exists x           | \neg \forall x \neg |           |
\end{lstlisting}



\subsection{First order logic}
\subsubsection{Some problems}
\paragraph{Student problems}
\begin{quote}
  \begin{enumerate}
  \item /Some students took French in spring 2001./
  \item /Every student who takes French passes it./
  \item /Only one student took Greek in spring 2001./
  \item /The best score in Greek is always higher than the best score in French./
  \end{enumerate}
\end{quote}

\begin{description}
\item [student(x)] x is student
\item [f,g] French and German courses
\item [take(x,c,s)] student =x= takes course =c= in semester =s=
\item [pass(x,c)] student =x= passes course =c=
\item [score(x,c,s)] the score of student =x= in course =c= in semester =s=.
\item [x>y] x is greater than y
\end{description}

\begin{quote}
  \begin{enumerate}
  \item $\exists x student(x) \wedge take(x,f,spring2001)$
  \item $\forall x,s student(s) \wedge take(x,f,s) \Rightarrow pass(x,f,s)$
  \item $\exists x student(x) \wedge take(x,g,sprint2001) \wedge \forall \: y y \ne x \Rightarrow \neg take(y,g,sprint2001)$
  \item $\forall s \exists x \forall y score(x,g,s) > score(y,f,s)$
  \end{enumerate}
\end{quote}

\paragraph{pollicy problems}

\begin{quote}
    5. /Every person who buys a policy is smart./
    6. /No person buys an expensive policy./
    7. /There is an agent who sells policies only to people who are not insured./
\end{quote}

\begin{description}
\item [person(x)] x is person
\item [expensive(x)] x is expensive
\item [agent(x)] x is agent
\item [insured(x)] x is insured
\item [smart(x)] x is smart
\item [buy(x,y,z)] =x= buys =y= from =z=
\item [sell(x,y,z)] =x= sells =y= to =z=
\end{description}

\begin{quote}
  \begin{enumerate}
  \item $\forall person(x) \wedge (\exists y,z policy(y) \wedge buy(x,y,z)) \Rightarrow smart(x)$
  \item $\forall x,y,z person(x) \wedge policy(y) \wedge expensive(y) \Rightarrow \neg buy(x,y,z)$
  \item $\exists x agent(x) \wedge \forall y,z policy(y) \wedge sell(x,y,z) \Rightarrow (person(z) \wedge \neg insured(z))$
  \end{enumerate}
\end{quote}

\paragraph{barber}
\begin{quote}
  8. /There is a barber who shaves all men in town who do not shave themselves./
\end{quote}
\begin{description}
\item [man(x)] x is man
\item [barber(x)] x is a barber
\item [shaves(x,y)] =x= shaves =y=
\end{description}
\begin{quote}
  8. $\exists x \forall y barber(x) \wedge man(y) \wedge \neg shaves(y,y) \Rightarrow shaves(x,y)$
\end{quote}


\paragraph{citizen}
\begin{quote}
    9. /A person born in the UK, each of whose parents is a UK citizen or a UK resident, is a UK citizen by birth./
    10. /A person born outside the UK, one of whose parents is a UK citizen by birth, is a UK citizen by descent./
\end{quote}
\begin{description}
\item [person(x)] x is person
\item [parent(x,y)] =x= is parent of =y=
\item [citizen(x,c)] =x= is a citizen of country =c=
\item [citizen(x,c,r)] =x= is a citizen of country =c=, for reason =r=
\item [resident(x,c)] =x= is a resident of country =c=
\item [born(x,c)] =x= was born in country =c=
\end{description}

\begin{quote}
  \begin{enumerate}
  \item
    $\forall x person(x) \wedge born(x,UK) \wedge (\forall y
    parent(y,x) \Rightarrow ((\exists r citizen(y,UK,r)) \vee
    resident(y,UK))) \Rightarrow citizen(x,UK,BIRTH)$
  \item
    $\forall x person(x) \wedge \neg born(x,UK) \wedge (\exists y
    parent(y,x) \wedge citizen(y,UK,BIRTH)) \Rightarrow
    citizen(x,UK,DESCENT)$
  \end{enumerate}
\end{quote}

\paragraph{other}
\begin{quote}
    11. /Politicians can fool some of the people all of the time,/
        /and they can fool all of the people some of the time,/
        /but they can’t fool all of the people all of the time./

    12. /All Greeks speak the same language. (Use Speaks(x,l) to mean that person x speaks language l.)/
\end{quote}
\begin{description}
\item [person(x)] x is person
\item [politician(x)] x is politician
\item [fool(x,y,t)] =x= fools =y= at time =t=
\item [german(x)] =x= is German.
\end{description}
\begin{quote}
  \begin{enumerate}
  \item
    $\forall x politician(x) \Rightarrow (\exists y \forall t
    person(y) \wedge fool(x,y,t)) \wedge (\exists t \forall y
    person(y) \Rightarrow fool(x,y,t)) \wedge \neg (\forall t \forall
    y person(y) \Rightarrow fool(x,y,t))$
  \item
    $\forall x,y,l german(x) \wedge german(y) \wedge Speaks(x,l)
    \Rightarrow Speaks(y,l)$
  \end{enumerate}
\end{quote}

\subsubsection{ Unify}
\begin{quote}
  /For each pair of atomic sentences, give the most general unifier if it exists:/
  \begin{enumerate}
  \item $P(A,B,B), P(x,y,z)$
  \item $Q(y,G(A,B)),Q(G(x,x),y)$
  \item $Older(Father(y),y),Older(Father(x),John)$
  \item $Knows(Father(y),y), Knows(x,x)$
  \end{enumerate}
\end{quote}

\paragraph{1} {x/A, y/B, z/B}

\begin{eqnarray}
    /P(A,B,B), P(x,y,z)/\\
    \Rightarrow /P(A,B,B), P(A,y,z)/ : {x/A}\\
    \Rightarrow /P(A,B,B), P(A,B,z)/ : {x/A, y/B}\\
    \Rightarrow /P(A,B,B), P(A,B,B)/ : {x/A, y/B, z/B}
\end{eqnarray}

\paragraph{2}
Cannot unify.
To see why:

\begin{eqnarray}
    Q(y,G(A,B)),Q(G(x,x),y) : {y/G(x,x)}\\
    \Rightarrow Q(G(x,x), G(A,B)), Q(G(x,x), G(x,x)) : {y/G(x,x), x/A}\\
    \Rightarrow Q(G(A,A), G(A,B)), Q(G(A,A), G(A,A))
\end{eqnarray}
In the last formula, A cannot be unified to B.

\paragraph{3}
{x/John, y/John}

\begin{eqnarray}
  /Older(Father(y),y),Older(Father(x),John)/\\
  \Rightarrow /Older(Father(John),John),Older(Father(x),John)/ : {y/John}\\
  \Rightarrow /Older(Father(John),John),Older(Father(John),John)/ : {y/John, x/John}
\end{eqnarray}

\paragraph{4}
cannot unify.
\begin{eqnarray}
  Knows(Father(y),y), Knows(x,x) : {x/Father(y)}\\
  \Rightarrow Knows(Father(y),y), Knows(Father(y),Father(y))
\end{eqnarray}

In the last formula, y cannot be unified to Father(y).

\subsubsection{Barber}
\begin{quote}
   /9.20 Let L be the first-order language with a single predicate S(p, q),/
   /meaning “p shaves q.” Assume a domain of people./

   \begin{enumerate}
   \item /Consider the sentence “There exists a person P who shaves every one who does not shave themselves,/
     /and only people that do not shave themselves.”/
     /Express this in L./
   \item /Convert the sentence in (a) to clausal form./
   \item /Construct a resolution proof to show that the clauses in (b) are inherently inconsistent./
     /(Note: you do not need any additional axioms.)/
   \end{enumerate}
\end{quote}

\paragraph{1}
$$\exists p \forall q person(p) \wedge person(q) \wedge (\neg S(q,q) \Leftrightarrow S(p,q))$$

\paragraph{2}
1st order formula:

\begin{equation}
  \exists p \: \forall q \: person(p) \wedge person(q) \wedge (\neg S(q,q) \Leftrightarrow S(p,q))
\end{equation}

\begin{equation}
  \exists p \: \forall q \: person(p) \wedge person(q) \wedge (\neg S(q,q) \Rightarrow S(p,q)) \wedge (S(p,q) \Rightarrow \neg S(q,q))
\end{equation}

remove implication:

\begin{equation}
  \exists p \: \forall q \: person(p) \wedge person(q) \wedge (S(q,q) \vee S(p,q)) \wedge (\neg S(p,q) \vee \neg S(q,q))
\end{equation}

skolemize off the existence:

\begin{equation}
  \forall q \: person(P) \wedge person(q) \wedge (S(q,q) \vee S(P,q)) \wedge (\neg S(P,q) \vee \neg S(q,q)) : \{p={P}\}
\end{equation}

drop universal qualifier:

\begin{equation}
  person(P) \wedge person(q) \wedge (S(q,q) \vee S(P,q)) \wedge (\neg S(P,q) \vee \neg S(q,q))
\end{equation}

\paragraph{3}

The CNF above resolves to empty clause, which is false, meaning the logic is not satisfiable.


\subsection{Bayesian}
\begin{itemize}
\item  Node X is conditionally independent of all other nodes in the
  network, given its markov blanket. (parents, children, and
  children's parents).
\item  Node X is conditionally independent of its non-descendants given its
  parent.
\end{itemize}

\begin{description}
\item [sample space] The set of all possible worlds
\item [$\Omega$] the sample space
\item [$\omega$] an element of the space. Each element has a probability
  $P(\omega)$, and sum up to one.
\item [prior] also called /unconditional probabilities/ or /prior
  probabilities/.
\item [posterior] also called /conditional probability/ or /posterior
  probability/.
\item [evidence] the results observed
\item [product rule] A different form of the definition of conditional probability
  $P(a\wedge b) = P(a|b) P(b)$
\item [random variables] begin with uppercase letter.
\item [domain] the set of possible values the random variable can take
\item [probability distribution] for discrete random variables
\item [probability density function] for continuous random variables,
  because the vector is infinite.
\item [joint probability distribution] the P for two variables with some
  interaction
\item [full joint probability distribution] all variables
\item [inclusion-exclusion principle]
  $P(a\vee b) = P(a) + P(b) - P(a\wedge b)$
\item [probabilistic inference] computation of posterior probabilities given evidence
\item [marginalization] also called /summing out/, because it sums the
  conditional probabilities. The process of computing the
  unconditional probability, aka marginal probability
\item [normalization] when calculating a conditional probability, there's
  a constant. It is typically not important, so replace it with
  $\alpha$. The alpha is to set the probabilities to sum to 1.
  $P(X|e) = \alpha P(X,e)$
\item [conditioning rule] $P(Y) = \sum_{z} P(Y|z)P(z)$
\item [(absolute) independence] $P(X|Y)=P(X)$
\item [Bayes rule] $P(b|a) = \frac{P(a|b)P(b)}{P(a)}$
\item [conditional independence] $P(X,Y|Z) = P(X|Z) P(Y|Z)$
\item [naive Bayes] a single cause directly influences a number of
  effects, all of which conditionally independent, given the cause.
  It is also called /Bayesian classifier/, /idiot Bayes/.
  $P(Cause,E1,E2,...,En) = P(Cause) \prod_i P(Ei|Cause)$
\item [conditional probability table (CPT)] each row shows a conditional probability
\item each variable is conditionally independent of its non-descendants, given its parents.
\item [Markov blanket] parents, children, and children's parents
\end{description}

For continuous variables, the Bayes needs to do something.  Of course
we can do discretization, but the precision is lost.  One common
solution is to define standard families of probability density
functions, with a finite number of parameters.
Another solution is non-parameter one.
Now I list some parameter models.

\begin{description}
\item [Gausion (normal) distribution] $N(\mu, \sigma^2)(x)$ has the mean $\mu$ and the variance $\sigma^2$.
\end{description}


A network with both discrete and continuous variables is called hybrid
Bayesian network.
\begin{description}
\item [linear Gaussian]
\item [conditional Gaussian]
\item [probit distribution]
\item [logit distribution]
\item [logistic function]
\end{description}

Exact Inference
\begin{description}
\item [query variables]
\item [event] some assignments to a set of evidence variables
\item [evidence variables]
\item [hidden variables] non-query non-evidence variables
\end{description}





%%% Local Variables:
%%% TeX-master: "cheatsheet"
%%% End:
