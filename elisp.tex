\documentclass[10pt,landscape]{article}
\usepackage{multicol}
\usepackage{calc}
\usepackage{ifthen}
\usepackage[landscape]{geometry}
\usepackage{amsmath,amsthm,amsfonts,amssymb}
\usepackage{color,graphicx,overpic}
\usepackage{hyperref}
\usepackage{listings}
\usepackage{textcomp}


\lstset{upquote=true}


\pdfinfo{
  /Title (example.pdf)
  /Creator (TeX)
  /Producer (pdfTeX 1.40.0)
  /Author (Seamus)
  /Subject (Example)
  /Keywords (pdflatex, latex,pdftex,tex)}

% This sets page margins to .5 inch if using letter paper, and to 1cm
% if using A4 paper. (This probably isn't strictly necessary.)
% If using another size paper, use default 1cm margins.
\ifthenelse{\lengthtest { \paperwidth = 11in}}
    { \geometry{top=.5in,left=.5in,right=.5in,bottom=.5in} }
    {\ifthenelse{ \lengthtest{ \paperwidth = 297mm}}
        {\geometry{top=1cm,left=1cm,right=1cm,bottom=1cm} }
        {\geometry{top=1cm,left=1cm,right=1cm,bottom=1cm} }
    }

% Turn off header and footer
\pagestyle{empty}

% Redefine section commands to use less space
\makeatletter
\renewcommand{\section}{\@startsection{section}{1}{0mm}%
                                {-1ex plus -.5ex minus -.2ex}%
                                {0.5ex plus .2ex}%x
                                {\normalfont\large\bfseries}}
\renewcommand{\subsection}{\@startsection{subsection}{2}{0mm}%
                                {-1explus -.5ex minus -.2ex}%
                                {0.5ex plus .2ex}%
                                {\normalfont\normalsize\bfseries}}
\renewcommand{\subsubsection}{\@startsection{subsubsection}{3}{0mm}%
                                {-1ex plus -.5ex minus -.2ex}%
                                {1ex plus .2ex}%
                                {\normalfont\small\bfseries}}
\makeatother

% Define BibTeX command
\def\BibTeX{{\rm B\kern-.05em{\sc i\kern-.025em b}\kern-.08em
    T\kern-.1667em\lower.7ex\hbox{E}\kern-.125emX}}

% Don't print section numbers
\setcounter{secnumdepth}{0}


\setlength{\parindent}{0pt}
\setlength{\parskip}{0pt plus 0.5ex}

%My Environments
\newtheorem{example}[section]{Example}
% -----------------------------------------------------------------------

\begin{document}
\raggedright
\footnotesize
\begin{multicols}{3}


% multicol parameters
% These lengths are set only within the two main columns
%\setlength{\columnseprule}{0.25pt}
\setlength{\premulticols}{1pt}
\setlength{\postmulticols}{1pt}
\setlength{\multicolsep}{1pt}
\setlength{\columnsep}{2pt}

\begin{center}
     \Large{\underline{Emacs Lisp Cheatsheet}} \\
\end{center}
























\section{Concepts}

\subsection{Quote}
\begin{tabular}{@{}ll@{}}
  \lstinline!'! & read without evaluate \\
  \lstinline!`! & same as \lstinline!'! except it can partially evaluate.\\
  \lstinline!,! & evaluate inside back quote\\
  \lstinline!,@! & evaluate and splice the resulting list inside back quote.
\end{tabular}

\subsection{Symbol}
Three special forms define symbols: \texttt{defvar}, \texttt{defun}, and \texttt{defmacro}.
A symbol can not be both a function and macro, but it can be a variable and a function at the same time.

A symbol can have a property list.
\begin{tabular}{@{}ll@{}}
  \texttt{get <sym> <prop>} & get\\
  \texttt{put <sym> <prop> <val>} & set\\
  \texttt{symbol-plist <sym>} & return the p-list\\
  \texttt{setplist <sym> <plist>} & set p-list
\end{tabular}


\section{IO}
\begin{tabular}{@{}ll@{}}
  \texttt{print} & in quote, with newline before and after\\
  \texttt{prin1} & in quote\\
  \texttt{princ} & no addition\\
  \texttt{message} & display at bottom, goes into \lstinline!*Messages*! buffer
\end{tabular}


\section{Regular Expression}
Text 2

\section{Type}
Etc.

\section{Function}

\section{Control Structure}








\end{multicols}
\end{document}
