\section{C}

\subsection{Type}
\begin{itemize}
\item When using \texttt{char}, specify whether you want to use
  \texttt{unsigned char} or \texttt{signed char}, otherwise the system
  will choose one.
\item \texttt{long long} >= \texttt{long} >= \texttt{int} >=
  \texttt{short}. So when choosing, always use \texttt{int} and
  \texttt{long long}
\item use \texttt{double} instead of \texttt{float}
\end{itemize}

\subsubsection{Static}
\begin{description}
\item [static member function] can use ~ClassName::function()~ directly
\item [static member variable] only one object for all instance of the
  class
\item [static variable] A static variable inside a function keeps its
  value between invocations.  A static global variable or a function
  is "seen" only in the file it's declared in
\item [static functions] Static functions are not visible outside of
  the C file they are defined in.
\end{description}

\begin{itemize}
\item outside any function, a variable of system type will be initialized to 0 if
\item inside a function, a variable of system type will \textit{not} be initialized.
\item string will be init to empty
\item \texttt{extern int i;} is declaration
\item \texttt{extern int i=1;} is definition, and cannot be inside a function
\end{itemize}

\subsection{Language Specification}
\subsubsection{variadic functions}
a function to take a variable number or type of arguments

Receiving of arguments:
You actually need both the number and type of the arguments to retrieve.

\begin{enumerate}
\item create \verb$va_list$
\item initialize using \verb$va_start$
\item access using multiple \verb$va_arg$. The first gives you the first arg, and so on.
\item call \verb$va_end$
\end{enumerate}

The macro prototypes (defined in =stdarg.h=). Note these are macros,
not functions.
\begin{lstlisting}
void va_start (va_list ap, last-required)
type va_arg (va_list ap, type)
void va_end (va_list ap)
\end{lstlisting}


An real world example:
\begin{lstlisting}
  #include <stdarg.h>
  #include <stdio.h>

  int
  add_em_up (int count,...)
  {
    va_list ap;
    int i, sum;
    va_start (ap, count);         /* Initialize the argument list. */
    sum = 0;
    for (i = 0; i < count; i++)
      sum += va_arg (ap, int);    /* Get the next argument value. */
    va_end (ap);                  /* Clean up. */
    return sum;
  }
  int
  main (void)
  {
    /* This call prints 16. */
    printf ("%d\n", add_em_up (3, 5, 5, 6));
    /* This call prints 55. */
    printf ("%d\n", add_em_up (10, 1, 2, 3, 4, 5, 6, 7, 8, 9, 10));
    return 0;
  }
\end{lstlisting}

\subsubsection{Variadic Macros}
\begin{lstlisting}
#define eprintf(format, ...) fprintf (stderr, format, __VA_ARGS__)
\end{lstlisting}

The \verb$VA_ARGS$ will be replaced with whatever in \verb$...$.
The variable argument is completely macro-expanded before it is inserted into the macro expansion,
just like an ordinary argument.

\subsection{idioms}
\begin{lstlisting}
typedef enum { false, true } bool;
\end{lstlisting}

\subsection{Cast}

\subsubsection{In a word}
\begin{description}
\item [static\_cast] ordinary type conversions.
\item [dynamic\_cast] converting pointers/references within an
  inheritance hierarchy.
\item [reinterpret\_cast] low-level reinterpreting of bit patterns.
  Use with extreme caution.
\item [const\_cast] casting away const/volatile.  Avoid this unless you
  are stuck using a const-incorrect API.
\end{description}

\subsubsection{C style cast}
DO NOT USE

\subsubsection{Static Cast}
\verb$static_cast$ is the first cast you should attempt to use.  It does
things like implicit conversions between types (such as =int= to
=float=, or pointer to =void*=), and it can also call explicit
conversion functions (or implicit ones).  In many cases, explicitly
stating \verb$static_cast$ isn't necessary, but it's important to note that
the ~T(something)~ syntax is equivalent to ~(T)something~ and should
be avoided (more on that later).  A \verb$T(something, something_else)$ is
safe, however, and guaranteed to call the constructor.

\verb$static_cast$ can also cast through inheritance hierarchies.  It is
unnecessary when casting upwards (towards a base class), but when
casting downwards it can be used as long as it doesn't cast through
virtual inheritance.  It does not do checking, however, and it is
undefined behavior to \verb$static_cast$ down a hierarchy to a type that
isn't actually the type of the object.

\subsubsection{Const Cast}

\verb$const_cast$ can be used to remove or add const to a variable; no
other C++ cast is capable of removing it (not even
\verb$reinterpret_cast$).  It is important to note that modifying a
formerly const value is only undefined if the original variable is
const; if you use it to take the const off a reference to something
that wasn't declared with const, it is safe.  This can be useful when
overloading member functions based on const, for instance.  It can
also be used to add const to an object, such as to call a member
function overload.

\verb$const_cast$ also works similarly on volatile, though that's less
common.

\subsubsection{Dynamic Cast}

\verb$dynamic_cast$ is almost exclusively used for handling polymorphism.
You can cast a pointer or reference to any polymorphic type to any
other class type (a polymorphic type has at least one virtual
function, declared or inherited).  You can use it for more than just
casting downwards -- you can cast sideways or even up another chain.
The \verb$dynamic_cast$ will seek out the desired object and return it if
possible.  If it can't, it will return NULL in the case of a pointer,
or throw \verb$std::bad_cast$ in the case of a reference.

\verb$dynamic_cast$ has some limitations, though.  It doesn't work if there
are multiple objects of the same type in the inheritance hierarchy
(the so-called 'dreaded diamond') and you aren't using virtual
inheritance.  It also can only go through public inheritance - it will
always fail to travel through protected or private inheritance.  This
is rarely an issue, however, as such forms of inheritance are rare.

\subsubsection{Reinterpret Cast}
\verb$reinterpret_cast$ is the most dangerous cast, and should be used very
sparingly.  It turns one type directly into another - such as casting
the value from one pointer to another, or storing a pointer in an int,
or all sorts of other nasty things.  Largely, the only guarantee you
get with \verb$reinterpret_cast$ is that normally if you cast the result
back to the original type, you will get the exact same value (but not
if the intermediate type is smaller than the original type).  There
are a number of conversions that \verb$reinterpret_cast$ cannot do, too.
It's used primarily for particularly weird conversions and bit
manipulations, like turning a raw data stream into actual data, or
storing data in the low bits of an aligned pointer.

\subsubsection{C style cast}

C casts are casts using (type)object or type(object).
A C-style cast is defined as the first of the following which succeeds:
\begin{description}
\item [const\_cast]
\item [static\_cast] (though ignoring access restrictions)
\item [static\_cast] (see above), then \verb$const_cast$
\item [reinterpret\_cast] \verb$reinterpret_cast$, then \verb$const_cast$
\end{description}

It can therefore be used as a replacement for other casts in some
instances, but can be extremely dangerous because of the ability to
devolve into a \verb$reinterpret_cast$, and the latter should be preferred
when explicit casting is needed, unless you are sure \verb$static_cast$
will succeed or \verb$reinterpret_cast$ will fail.  Even then, consider the
longer, more explicit option.

C-style casts also ignore access control when performing a
\verb$static_cast$, which means that they have the ability to perform an
operation that no other cast can.  This is mostly a kludge, though,
and in my mind is just another reason to avoid C-style casts.

\subsection{Compound Literals}

A compound literal looks like a cast containing an initializer.  Its
value is an object of the type specified in the cast, containing the
elements specified in the initializer; it is an lvalue.

Example

\begin{lstlisting}
  struct foo {int a; char b[2];} structure;
  // The constructing:
  structure = ((struct foo) {x + y, 'a', 0});
  char **foo = (char *[]) { "x", "y", "z" };
\end{lstlisting}

static Value in the compound literals must be constant.

\begin{lstlisting}
static struct foo x = (struct foo) {1, 'a', 'b'};
static int y[] = (int []) {1, 2, 3};
static int z[] = (int [3]) {1};
\end{lstlisting}

\subsection{extern}
\begin{itemize}
\item extern means extend the visibility of a variable or function.
\item Declaration can be many times, but definition can only appear once.
\item Definition will allocate memory, but declaration will never allocate memory.
\end{itemize}

\subsubsection{Function}

For function declare and define, `extern` is added by compiler by
default.  So use or not use `extern` for functions are equivalent.

\subsubsection{Variable}

define a variable

\begin{lstlisting}
int a;
\end{lstlisting}

declare a variable

\begin{lstlisting}
extern int a;
\end{lstlisting}

This can be used so that in this file, a refer to the variable
actually defined and allocated in another file.  The definition of the
variable in the other file does not have extern, but it is still
available by this file ...

An exception: extern a variable with initialization

\begin{lstlisting}
extern int a = 8;
\end{lstlisting}

This will be treated as definition.

\subsubsection{extern "C"}
\begin{quote}
extern "C" makes a function-name in C++ have 'C' linkage
(compiler does not mangle the name)
so that client C code can link to (i.e use) your function
using a 'C' compatible header file
that contains just the declaration of your function.
\end{quote}

\begin{enumerate}
\item Since C++ has overloading of function names and C does not
\item C++ compiler cannot just use the function name as a unique id to link to, so it mangles the name by adding information about the arguments
\item A C compiler does not need to mangle the name since you can not overload function names in C
\end{enumerate}

When you state that a function has extern "C" linkage in C++,
the C++ compiler does not add argument/parameter type information
to the name used for linkage.

syntax
\begin{itemize}
\item can specify "C" linkage to each individual declaration/definition explicitly
\item use a block to group a sequence of declarations/definitions to have a certain linkage:
\end{itemize}
\begin{lstlisting}
extern "C" void foo(int);
extern "C"
{
   void g(char);
   int i;
}
\end{lstlisting}


\subsection{restrict}
The restrict keyword is a declaration of intent given by the programmer to the compiler.

It says that for the lifetime of the pointer,
only it or a value directly derived from it (such as pointer + 1)
will be used to access the object to which it points.

This limits the effects of pointer aliasing, aiding optimizations.

If the declaration of intent is not followed
and the object is accessed by an independent pointer,
this will result in undefined behavior.



\subsection{volatile}
When your code works without compiler optimization, but fails when you turn optimization on,
perhaps it is because of `volatile`.

If compiler found that around a variable, no one change it, it will do some optimization based on this.
Maybe remove unnecessary code which it thinks will never execute.

The keyword tells the compiler that the value of the variable may change at any time.
It may change unexpectedly,
so DO NOT optimize the code when you compiler think it would not change.

\subsubsection{syntax}

declare a variable(both are equalvalent)

\begin{lstlisting}
volatile int foo;
int volatile foo;
\end{lstlisting}

declare pointers to volatile varialbes(common usage)

\begin{lstlisting}
volatile uint8_t *pReg;
uint8_t volatile *pReg;
\end{lstlisting}

volatile pointers to non-volatile data(very rare)

\begin{lstlisting}
int * volatile p;
\end{lstlisting}

volatile pointer to volatile variable(also rare)

\begin{lstlisting}
int volatile * volatile p;
\end{lstlisting}

\subsubsection{When to use it}

\paragraph{Memory-mapped peripheral registers}
The register's value may change by hardware.  But in the code,
compiler cannot see it, so it may assume it is constant, and do some
optimization.

\begin{lstlisting}
uint8_t *pReg = (uint8_t) 0x1234;
while (*pReg==0) {}
\end{lstlisting}

Since no `volatile`, the assembly looks like:

\begin{lstlisting}
  mov ptr, #0x1234
  mov a, @ptr
loop:
  bz loop
\end{lstlisting}

To fix it, use volatile to declare it:

\begin{lstlisting}
uint8_t volatile *pReg = (uint8_t volatile *)0x1234
\end{lstlisting}

The assembly will be:

\begin{lstlisting}
  mov ptr, #0x1234
loop:
  mov a, @ptr
  bz loop
\end{lstlisting}
\paragraph{Global variables modified by an ISR(Interrupt Service Routine)}
Compiler will of course not know about interrupt.
So when the global file can be modified by interrupt,
we must tell it.

\begin{lstlisting}
int volatile etx_rcvd = FALSE;
void main() {
  while(!ext_rcvd) {}
}
interrupt void rx_isr(void) {
  if (ETX == rx_char) {
    etx_rcvd = TRUE;
  }
}
\end{lstlisting}

If no volatile, compiler will think the while condition always be true,
thus never go out of the loop.

\paragraph{Global variables accessed by multiple tasks within a multi-threaded application}
Compiler doesn't find the variable change near the code it is defined,
so it may assume it is unchanged.
While another task in the same time may change it,
it is just like the interrupt.

\subsection{Operator Precedence}

\begin{table}[ht]
  \centering
  \tiny
  \begin{tabular}{r|l|l|l}
    Pred & Operator & Description & Assoc\\
    \hline
    0 & \textbf{\textbf{::}} & scope resolution & L to R\\
    \hline
    1 & ++ -- & Suffix increment and decrement & \\
               & () & Function call & \\
               & [] & Array subscripting & \\
               & . & Structure and union member access & \\
               & -> & Struct/union member access through pointer & \\
               & (type)\{list\} & Compound literal(C99) & \\
    \hline
    2 & ++ -- & Prefix increment and decrement & R to L\\
               & + - & Unary plus and minus & \\
               & ! \textasciitilde{} & Logical NOT and bitwise NOT & \\
               & (type) & Type cast & \\
               & * & dereference & \\
               & \& & Address-of & \\
               & sizeof &  & \\
               & \_Alignof & Alignment requirement(C11) & \\
               & \textbf{\textbf{new, new[]}} & Dynamic memory allocation & \\
               & \textbf{\textbf{delete, delete[]}} & Dynamic memory deallocation & \\
    \hline
    3 & * / \% &  & L to R\\
    4 & + - & Addition and subtraction & \\
    5 & << >> & Bitwise left shift and right shift & \\
    6 & < <= & Compare & \\
               & > >= &  & \\
    7 & \texttt{= !} &  & \\
    8 & \& & Bitwise AND & \\
    9 & \^{} & Bitwise XOR (exclusive or) & \\
    10 & l & Bitwise OR (inclusive or) & \\
    11 & \&\& & Logical AND & \\
    12 & ll & Logical OR & \\
    13 & ?: & Ternary conditional & R to L\\
    \hline
    14 & \textbf{\textbf{throw}} &  & \\
               & = &  & \\
               & += -= &  & \\
               & *= /= \%= &  & \\
               & <<= >>= & Assignment by bitwise left shift and right shift & \\
               & \&= \^{}= l= & Assignment by bitwise AND, XOR, and OR & \\
    \hline
    15 & , & Comma & L to R\\
  \end{tabular}
  \caption{Precedence}
\end{table}

\subsubsection{notes}
\begin{itemize}
\item for \texttt{?:}: the middle of the conditional operator (between ? and :) is
  parsed as if parenthesized: its precedence relative to =?:= is
  ignored
\item For C++: The operand of sizeof can't be a C-style type cast:
 the expression =sizeof (int) * p= is unambiguously interpreted as =(sizeof(int)) * p=,
 but not =sizeof((int)*p)=.
\item In c++ table, the =?:= is also in 14 cell
\end{itemize}

\subsection{Unix Library}
sleep
\begin{lstlisting}
#include <unistd.h>
unsigned int sleep(unsigned int seconds); // seconds
int usleep(useconds_t useconds); // microseconds
int nanosleep(const struct timespec *rqtp, struct timespec *rmtp);
\end{lstlisting}

There's no implementation of \verb$clock_gettime$ as in Unix, the following serves as an portable solution.
\begin{lstlisting}
#ifdef __MACH__
#include <sys/time.h>
#define CLOCK_REALTIME 0
#define CLOCK_MONOTONIC 0
//clock_gettime is not implemented on OSX
int clock_gettime(int /*clk_id*/, struct timespec* t) {
    struct timeval now;
    int rv = gettimeofday(&now, NULL);
    if (rv) return rv;
    t->tv_sec  = now.tv_sec;
    t->tv_nsec = now.tv_usec * 1000;
    return 0;
}
#endif
\end{lstlisting}

use it like this
\begin{lstlisting}
double get_time() {
  struct timespec ts;
  ts.tv_sec=0;
  ts.tv_nsec=0;
  clock_gettime(CLOCK_REALTIME, &ts);
  double d = (double)ts.tv_sec + 1.0e-9*ts.tv_nsec;
  return d;
}
\end{lstlisting}

%%% Local Variables:
%%% TeX-master: "cheatsheet"
%%% End:
