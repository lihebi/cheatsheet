\section{R}

\subsection{Setting up}
\subsubsection{Packages}
\begin{description}
\item [library()] see the list of installed packages.
\item [library(class)] load the package =class=.
\item [search()] see the list of loaded packages.
\item [install.packages()] and =update.packages()= install and update packages.
\end{description}

\subsection{Interface}
To start ESS session, run command \texttt{R}.

\subsubsection{ESS bindings}
\begin{description}
\item [C-c C-c] ess-eval-region-or-function-or-paragraph-and-step
\item [C-c C-r] ess-eval-region
\end{description}

\subsubsection{Babel}
There's a babel in org mode for R, so just \texttt{C-c C-c} would
work.  This will prompt to create a session.  One special for this
babel is you can evaluate by line, using \texttt{C-<RET>} in the edit
buffer.

\begin{description}
\item [\#+PROPERTY: session *R*] Keep the session
\item [:results output graphics] export a graph
\item [:results output org] summary data in org format
\item [:file a.png] write the graph into a file
\item [:exports both] export both code and result
\end{description}



\subsubsection{commands}
\begin{description}
\item [setwd("/absolute/path")] set the working directory.
\item [q()] to exit, 
\item [help(lm)] to show help message for a built-in function.
\item [?lm] shows documentation
\item [??solve] shows documentation
\item [example(topic)] shows the examples.
\item [help.start()] opens the html documentation page.
\item [source("a.R")] loads a script.
\item [sink("a.lis")] redirects the output to a file, and =sink()=
  restore that to standard output.
\item [ls()] list the objects currently stored.
\item [objects()] create and store current objects.
\item [rm(x, y, z, ink, junk, temp, foo, bar)] remove objects
\end{description}

\subsection{Objects}
Objects have mode and length.
\begin{description}
\item [Vectors] are the array of objects of the same mode.
\item [List] can contain the objects of different modes.
\item [data frame] lists of objects all have the same length
\item [factor] enumerators
\end{description}

The missing values are \texttt{NA}, tested by \texttt{is.na}.  Illegal
computations produces \texttt{NaN}, e.g. \texttt{0/0}.

\subsubsection{Get Attributes}
\begin{description}
\item [typeof] gets the type of an object
\item [mode] retrieves the mode of an object.
\item [length] get the length of vector
\item [names] is used for indexing. Data frame has names as column
  names. Assign a vector to the function will set the names.
\item [dim] is the dimension of a matrix. Setting dim of a variable
  declares a matrix. \texttt{dim(z) <- c(3,5)}.
\item [dimnames] is the character names of the dimensions
\item [str] show the structure of arbitrary type
\end{description}

\subsubsection{Data Frame}
A data frame is a list of equal-length vectors.  When getting the data
from \texttt{read.csv}, the result is a data frame.

Indexing:

\begin{description}
\item [{[]}] Return data frame, retaiing names. Can use vector as index.
\item [{[[]]}] Return the value, dropping names. Can NOT use a vector as index.
\end{description}

\subsubsection{Vector}
\begin{description}
\item [c(1,2,3)] creating a vector
\item [1:3] same as \texttt{c(1,2,3)}
\item [3:1] same as \texttt{c(3,2,1)}
\item [2*1:3] same as \texttt{c(2,4,6)}, \texttt{:} has higher priority
\item [seq(1,3)] produces \texttt{1:3}. More powerful
\item [seq(-5, 5, by=.2)]
\item [seq(length=51, from=-5, by=.2)]
\item [rep(c(1,2), times=2)] \verb$[1,2,1,2]$
\item [rep(c(1,2), each=2)] \verb$[1,1,2,2]$
\end{description}

\subsubsection{Evaluation Rules}
\begin{description}
\item [recycling rules] the shortest list is recycled to the length of longest.
\item [dimensional attributes] the dimension of matrix must match. No recycle for a matrix.
\end{description}


\subsubsection{Indexing}
\begin{description}
\item [{[]}] empty index will returns the entire vector
  \begin{itemize}
  \item with \textit{irrelevant} attributes removed.
  \item The only retained ones are the \texttt{names}, \texttt{dim}
    and \texttt{dimnames} attributes.
  \item Similarly the empty field in a matrix index will select the
    whole line or column. \texttt{z[2,,]}
  \end{itemize}
\item [{[1]}] using number as index.
  \begin{itemize}
  \item Start from 1
  \item Negative means elements other than those.
  \item The index 0 will return empty
  \item Other numeric values will be converted to integer towards
    zero.
  \end{itemize}
\item [{[c(True, False)]}] logical vector, the true ones will be
  returned.
\item [{[c("name")]}] character
  \begin{itemize}
  \item it is compared, partially, with \texttt{names} attributes of
    the vector
  \item can be a number or character, or a vector of them.
  \item \texttt{\$} can be used for indexing with character.
  \end{itemize}
\item [vector{[]}] return element
\item [list{[]}] return list
\item [list{[[]]}] return single element. Cannot use vector as index.
\end{description}

\subsection{Functions}
Useful R functions
\begin{description}
\item [na.omit] can omit the NA values in data frame
\item [commandArgs] return a list of argument. \texttt{csvfile=args[1]}
  \begin{description}
  \item [trailingOnly] TRUE or FALSE
  \end{description}
\item [subset] return a subset by some
  criteria. \texttt{subset(csv,lastcol>=5)}. Condition can be
  concatenated by \verb$&$ and \verb$|$
\item [substr] \texttt{substr(name, 1, 6)=="output"}
\item [paste] takes an arbitrary number of arguments and concatenates
  them one by one into character strings.
\item [read.csv] return a data frame.
  \begin{description}
  \item [header] TRUE or FALSE
  \end{description}
\item [write]
\item [write.table]
\item [write.csv]
\item [read.table("filename", header=TRUE, sep=",")]
  \begin{itemize}
  \item this ignores blank lines,
  \item and expect the header to be one field less than the body.
  \item \verb$#$ as comments
  \end{itemize}
\end{description}

\begin{tabular}{llll}
  \verb$^$ & exp & \verb$%%$ & modulus\\
  \verb$%*%$ & matrix product & \verb$%o%$ & outer product\\
  \verb$!$ & negative & \verb$&,|$ & for vector\\
  \verb$&&,||$ & for no vector & \verb$<-, ->$ & assignments\\
  \texttt{\$} & list subset & \verb$:$ & sequence\\
  \verb$~$ & for model formula
\end{tabular}

Mathmatic functions
\begin{description}
\item [log, exp]
\item [sin,cos,tan]
\item [sqrt]
\item [min,max]
\item [range] same as \texttt{c(min(x),max(x))}
\item [sum,prod]
\item [mean] \texttt{sum(x)/length(x)}
\item [var(x)] \verb$sum((x-mean(x))^2)/(length(x)-1)$
\item [sort] increasing order
\item [order] \texttt{sort.list}
\end{description}


\subsection{Control Structure}
The compound statements are the same as C, can be a single statement
without the braces.

\subsubsection{Condition}
Condition can use \texttt{==} to check. String can also be checked
this way. \texttt{!} can be used to negate a whole boolean
vector. \texttt{!is.na(col)}.

\subsubsection{Conditional}

\begin{description}
\item [if (\NT{stmt}) \NT{stmt} else if (\NT{stmt}) \NT{stmt} else \NT{stmt}]
\item [switch (\NT{stmt}, \NT{list})] the \texttt{stmt} is first
  evaluated, if the value is within \texttt{[1,length(list)]} evaluate
  \texttt{list[i]}
\end{description}

\subsubsection{Loop}
\begin{description}
\item [for (\NT{name} in \NT{vector}) \NT{stmt}]
\item [while (\NT{stmt}) \NT{stmt}]
\item [repeat \NT{stmt}]
\item [break, next]
\end{description}

\subsection{Function}
\begin{description}
\item [function (\NT{arglist}) \NT{body}] The argument list can be a
  symbol, a \texttt{symbol=value}, or a \texttt{...}.  The body is a
  compound expression, surrounded with \verb${}$.  Function can be
  assigned to a symbol.
\end{description}

The matching of formals and actual are pretty tricky.
A return can return value from a function.
\begin{enumerate}
\item exact matching on tags
\item partial matching on tags
\item positional matching for \texttt{...}
\end{enumerate}
Partial matching result must be unique, but the exact matched ones are
excluded before this step is entered.

\subsection{Quote}
The quote will wrap the expression into an object without evaluating it.
The resulting object has the mode of =call=.
The =eval= is used to evaluate it.
\subsection{Debugging}
To enter the debugger, a call to =browser= function suffices.
This allows you to browse the values at that point.
A more powerful debugger is by a call to =debug= with the function name as argument.
Each time that function is called, you enter the debug and can control the execution.
Tracing can be registered by =trace= or =untrace= with the name of the function.
It might need to be quoted in some case, so you'd better quote it, with double quotes.
Every time the function is invoked, the return value will be printed as trace.

\subsection{Models}
Linear model.

\begin{lstlisting}
fm = lm(y ~ x1 + x2, data = mydataframe)
\end{lstlisting}

The fitted model in the variable =fm= can be accessed by:
\begin{description}
\item [coef] extract the coefficients
\item [deviance] the Residual Sum of Square
\item [formula] extract the model formula
\item [plot] produce four plots: residuals, fitted values, diagnostics.
\item [predict(OBJECT, newdata=DATA.FRAME)] use the model to predict
\item [residuals] extract the residuals
\item [summary] print the summary
\end{description}

The models can be updated, if the formula only changes a little bit.
In the following example, the =.= means the corresponding part of the
original formula.
\begin{lstlisting}
fs <- lm(y~x1 + x2, data=mydata)
fs <- update(fs, . ~ . + x3)
fs <- update(fs, sqrt(.) ~ .)
\end{lstlisting}

\subsection{SVM}
\begin{lstlisting}
#!/usr/bin/env Rscript


# load the 1071 library containing vm
library("e1071")

# preview the data set
head(iris, 5)
attach(iris)
x <- subset(iris, select=-Species)
y <- Species

# using svm to get the model
svm_model <- svm(Species ~ ., data=iris)
summary(svm_model)

# another way to get the model
svm_model1 <- svm(x,y)
summary(svm_model1)

# predict using the same data
pred <- predict(svm_model1,x)
system.time(pred <- predict(svm_model1,x))

# table will give the precision and recall
table(pred,y)

error <- y - pred
svrPredictionRMSE <- rmse(error)

# using the "tune" function to automatic select the best using cross validation
svm_tune <- tune(svm, train.x=x, train.y=y, 
              kernel="radial", ranges=list(cost=10^(-1:2), gamma=c(.5,1,2)))
print(svm_tune)

## use the cost and gamma parameter got from tune,
## re-generate the model
svm_model_after_tune <-
    svm(Species ~ ., data=iris, kernel="radial", cost=1, gamma=0.5)
summary(svm_model_after_tune)

# re-run the prediction
pred <- predict(svm_model_after_tune,x)
system.time(predict(svm_model_after_tune,x))
# see the result
table(pred,y)
\end{lstlisting}

\subsection{Packages}
\subsubsection{ggplot2}
\begin{lstlisting}
qplot(totbill, tip, geom="point", data=tips) # scatter plot
qplot(totbill, tip, geom="point", data=tips) + geom_smooth(method="lm") # with linear relationship line
qplot(tip, geom="histogram", data=tip) # histogram
qplot(tip, geom="histogram", binwidth=1, data=tips) # with custom binwidth
# box plots
qplot(sex, tipperc, geom="boxplot", data=tips)
qplot(smoker, tipperc, geom="boxplot", data=tips)
qplot(sex:smoker, tipperc, geom="boxplot", data=tips) # combine! plot the two sets of graph in two one graph
qplot(totbill, tip, geom="point", colour=day, data=tips) # scatter plot with colors, in regard to "day" column
\end{lstlisting}

\subsubsection{plot(x, y, ...)}
Possible =...= arguments:
\begin{description}
\item [type] what type of plot:
  \begin{description}
  \item [p] for points,
  \item [l] for lines,
  \item [b] for both,
  \item [h] for =histogram= like (or =high-density=) vertical lines,
  \end{description}
\item [main] an overall title for the plot: see =title=.
\item [xlab] a title for the x axis: see =title=.
\item [ylab] a title for the y axis: see =title=.
\end{description}

\subsubsection{dplyr}
A Grammar of Data Manipulation
\begin{itemize}
\item https://cran.r-project.org/web/packages/dplyr/index.html
\item https://cran.rstudio.com/web/packages/dplyr/vignettes/introduction.html
\end{itemize}





%%% Local Variables:
%%% TeX-master: "cheatsheet"
%%% End:
