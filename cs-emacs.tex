\section{Emacs}

\subsection{general}
\begin{description}
\item [list-face-displays]
\item [fill-region]
\item [list-package] open the \texttt{*package*} buffer, \texttt{Ux}
  to update all.
\item [helm-etags-select]
\end{description}

\subsection{help}
\begin{description}
\item [describe-key-briefly] \texttt{C-h c} command name for a key
  stroke.
\item [where-is] \texttt{C-h w}, shortcut for a command
\item [info] \texttt{C-h i} open the built-in info reader.
\item [view-echo-area-messages]
\end{description}

\subsection{Dired}
\begin{description}
\item [dired-next-subdir]
\item [dired-prev-subdir]
\item [dired-tree-up]
\item [dired-tree-down]
\end{description}

\subsection{window}
\begin{description}
\item [balance-window]
\item [toggle-window-split]
\end{description}

\subsection{file}
\begin{description}
\item [revert-buffer] Reload file on disk
\item [recover-file] recover from \texttt{\#xxx\#} file.
\item [read-only-mode] disable it to edit read only files
\end{description}

\subsection{edit}

\begin{description}
\item [replace-rectangle]
\item [upcase-word]
\item [downcase-word]
\item [transpose-words]
\item [transpose-lines]
\item [C-q xxx] insert a control sequence
\item [capitalize-word]
\item [fill-paragraph]
\item [fill-region]
\item [auto-fill-mode]
\item [zap-to-char]
\item [zap-up-to-char]
\item [ispell] check spell 
\item [flycheck] check on-the-fly
\end{description}

\subsection{programming}

\begin{description}
\item [checkdoc] check the warnings in doc string
\item [C-x C-e] evaluate
\item [C-u C-x C-e] evaluate and insert result
\item [C-M-a] move to the beginning of defun
\item [C-M-e] move to the end of defun
\item [C-M-h] mark defun
\item [C-M-f] move forward a sexp
\item [C-M-b] move backward a sexp
\item [C-M-k] kill a sexp
\item [C-M-t] transpose expressions
\item [C-M-<SPC>] mark following sexp
\item [C-M-n] move to the next sexp
\item [C-M-p] move to the previous sexp
\item [C-M-u] move up parenthesis
\item [C-M-d] move down parenthesis
\item [C-M-x] evaluate defun
\item [forward-sexp] forward semantic block
\item [backward-sexp]
\item [org-forward-heading-same-level] =C-c C-f=
\item [org-backword-heading-same-level] =C-c C-b=
\end{description}

\subsection{Marking}
\begin{description}
\item [exhange-point-and-mark]
\item [mark-word]
\item [mark-sexp]
\item [mark-paragraph]
\item [mark-defun]
\item [mark-page]
\item [mark-whole-buffer]
\item [set-mark-command] C-SPC, set mark, and activate it
\item [C-SPC C-SPC] set mark, but not activate it.
\item [C-u C-SPC] pop to previous mark in mark ring. current is stored
  at the end of mark ring(rotating)
\item [pop-global-mark] will store both position and buffer
\item [set the mark] \texttt{C-SPC C-SPC}, \texttt{C-w}, search
\end{description}

\subsection{Register}
\begin{description}
\item [point-to-register] save ppposition in a register
\item [jump-to-register]
\item [jump-to-register] the register can store a file
\item [copy-to-register]
\item [insert-register]
\end{description}


\subsection{Key Map}
To configure a specific key map.  Note that the =global-set-key= will
\textit{not} overwrite a specific key map, because the specific one has a
higher priority.

\begin{lstlisting}[language=lisp]
  (define-key org-mode-map (kbd "C-j")
    (lambda()
      (interactive)
      (join-line -1)))
\end{lstlisting}

\subsection{Variables}
\paragraph{File Local Variable}
Emacs looks for the following line on first line or second line if
first line is shebang. \texttt{mode} defines the major mode for this
file, while unlimited number of variables follows, separated by
\texttt{;}.

\begin{lstlisting}
-*- mode: Lisp; fill-column: 75; comment-column: 50; -*-=
\end{lstlisting}


A second way to specify file local variable is to have a ``local
variables list'' near the end of the file (no more than 3000
characters from the end of the file).  The =LocalXXX Variables:= and
=XXXEnd:= will be matched literally, so literal that I have to add XXX
to these $\ldots$
\begin{lstlisting}
This     /* Local XXX Variables:  */
Is       /* mode: c           */
Garbage  /* comment-column: 0 */
AuCTex   TeX-master: "cheatsheet"
Data     /* XXXEnd:           */
\end{lstlisting}

\paragraph{Directory Local Variable}
Put \texttt{.dir-locals.el} at the root directory, and it will be in
effect for all the files under that directory, recursively.  It should
be an associate list, the car can be either a mode name (or \texttt{nil}
applies to all modes) indicating the variables are for that mode, or a
sub-directory name to apply only in that directory.
\begin{lstlisting}[language=lisp]
((nil . ((indent-tabs-mode . t)
         (fill-column . 80)))
 (c-mode . ((c-file-style . "BSD")
            (subdirs . nil)))
 ("src/imported"
  . ((nil . ((change-log-default-name
              . "ChangeLog.local"))))))
\end{lstlisting}


\subsection{Info}
Info is a document system.  It is closely bundled with emacs, so I put
it here.  To install some new info document in the system, issue the
following commands (using =gnu-c-manual= as an example):

\begin{lstlisting}
# download the gnu-c-manual code
make gnu-c-manual.info
mv gnu-c-manual.info /usr/local/share/info
cd /usr/local/share/info
sudo install-info --info-file=gnu-c-manual.info --info-dir=.
\end{lstlisting}

\subsubsection{Info Operations}

\begin{description}
\item [SPC] page down, can cross node
\item [BACKSPACE] page up, can cross node
\item [M-n] \texttt{clone-buffer}, open a new info buffer
\item [n] next node on same level
\item [p] previous
\item [{[}] next node regardless of level
\item [{]}] previous
\item [u] up node
\item [l] back
\item [r] forward
\item [m] ~Info-menu~, convenient for search node title
\item [s] TODO search  a text in the whole info file
\item [i] TODO search indices only
\end{description}


\subsection{YASnippet}
Put this into latex-mode directory as a file named ``desc'', and run \texttt{yas-reload-all}.
\begin{lstlisting}[language=tex]
# -*- mode: snippet -*-
# name: desciption
# key: desc
# --
\begin{description}
$1
\end{description}
$2
\end{lstlisting}
% \subsection{Babel}
% How to write a \texttt{ob-xxx.el} file?

% \begin{enumerate}
% \item search org-mode babel, you will get a link:
%   http://orgmode.org/worg/org-contrib/babel/
% \item In this link, there's a "languages"
%   link. http://orgmode.org/worg/org-contrib/babel/languages.html
% \item Under "Develop support for new languages" section, there's link
%   to ob-template.el:
%   http://orgmode.org/w/worg.git/blob/HEAD:/org-contrib/babel/ob-template.el
% \item follow instruction to modify it.
% \end{enumerate}
% some good example to look at: ob-plantuml.el, ob-C.el


\subsection{Magit}

\begin{description}
\item [C-x g] open magit interface
\item [F u] pull from upstream
\item [k] discard change
\item [s] stage
\item [u] unstage
\item [c c] commit. Then \texttt{C-c C-c} to submit commit
\item [P u] push upstream
\end{description}


%%% Local Variables:
%%% TeX-master: "cheatsheet"
%%% End:
