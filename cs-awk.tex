\section{AWK}
\subsection{Usage}


In shell, awk can be invoked like below. MUST USE single quote.

\begin{lstlisting}
awk 'pattern {action}'
awk 'pattern {action}' input.txt
awk -f script.awk input.txt
\end{lstlisting}

The form is \verb$pattern {action}$
A missing action means print the line;
a missing pattern always matches.
Pattern-action statements are separated by newlines or semicolons.

\subsection{Variables}
\begin{description}
\item [FS] \textit{Field Separator}. regular expression used to
  separate fields; also settable by option \texttt{-Ffs}.
\item [NF] number of fields in the current record
\item [NR] ordinal number of the current record
\item [FNR] ordinal number of the current record in the current file
\item [FILENAME] the name of the current input file
\item [RS] input record separator (default newline)
\item [OFS] output field separator (default blank)
\item [ORS] output record separator (default newline)
\end{description}


\subsection{patterns}

Patterns are arbitrary Boolean combinations (with \verb$!$= \verb$||$ \verb$&&$)
of regular expressions and relational expressions.

A pattern may consist of two patterns separated by a comma; in this
case, the action is performed for all lines from an occurrence of the
first pattern though an occurrence of the second.

special patterns: \texttt{BEGIN}, \texttt{END}.
Cannot combine with other patterns.

\subsection{actions}

An action is a sequence of statements.
Statements are terminated by semicolons, newlines or right braces

\subsection{variable}
awk is line-based, and the content of lines are splited into fields
\texttt{\$1}, \texttt{\$2}, etc, by the separator \texttt{FS}.
\texttt{\$0} refers to the entire line.

\subsection{examples}

\begin{lstlisting}
awk 'NR==10 {print}' input.txt # output 10th line, or empty is less than 10 lines
\end{lstlisting}


%%% Local Variables:
%%% TeX-master: "cheatsheet"
%%% End:
