\section{Elisp}
\subsection{Concepts}

\subsubsection{Quote}
\begin{tabular}{@{}ll@{}}
  \lstinline!'! & read without evaluate \\
  \lstinline!`! & same as \lstinline!'! except it can partially evaluate.\\
  \lstinline!,! & evaluate inside back quote\\
  \lstinline!,@! & evaluate and splice the resulting list inside back quote.
\end{tabular}

\subsubsection{Symbol}
Three special forms define symbols: \texttt{defvar}, \texttt{defun}, and \texttt{defmacro}.
A symbol can not be both a function and macro, but it can be a variable and a function at the same time.

A symbol can have a property list.

\begin{tabular}{@{}ll@{}}
  \texttt{get <sym> <prop>} & get\\
  \texttt{put <sym> <prop> <val>} & set\\
  \texttt{symbol-plist <sym>} & return the p-list\\
  \texttt{setplist <sym> <plist>} & set p-list
\end{tabular}


\subsection{IO}
\begin{tabular}{@{}ll@{}}
  \texttt{print} & in quote, with newline before and after\\
  \texttt{prin1} & in quote\\
  \texttt{princ} & no addition\\
  \texttt{message} & display at bottom, goes into \lstinline!*Messages*! buffer
\end{tabular}


\subsection{Regular Expression}
\subsubsection{Syntax}
\paragraph{Basics}
\begin{tabular}{@{}ll@{}}
  \texttt{.} & everything except newline\\
  \texttt{*,+,?} & \\
  \texttt{*?,+?,??} & non-greedy, match the smallest part\\
  \texttt{} &\\
  \texttt{\lstinline![],[^],^,\$!} &
\end{tabular}


\paragraph{Character classes}
Use the double \texttt{[[]]} form. E.g. \texttt{[[:ascii:]]}, and also the negative writes \lstinline![^[:ascii:]]!.

\begin{tabular}{@{}ll@{}}
  \texttt{ascii} & 0-127\\
  \texttt{alnum} & letter or digit\\
  \texttt{alpha} & letter\\
  \texttt{blank} & space and tab\\
  \texttt{cntrl} & ASCII control char\\
  \texttt{digit} & 0-9\\
  \texttt{lower} & lower-case\\
  \texttt{upper} & upper-case\\
  \texttt{punct} & punctuation\\
  \texttt{space} & whitespace\\
  \texttt{word} & same as \lstinline!\w!
\end{tabular}

\paragraph{Backslash}
Backslash is also used in elisp read syntax. So when you want one \texttt{\\} in the string, you need to write double.

\begin{tabular}{@{}ll@{}}
  \lstinline!\w, \W, \b, \B! &\\
  \lstinline!\{\}! & \lstinline!{}! counter part\\
  \lstinline!\1! & either one of two expressions. Can be used in capturing group\\
  \lstinline!\(\)! & capturing group\\
  \lstinline!\(?:\)! & non-capturing group\\
  \lstinline!\(?NUM:\)! & numbered capturing group.. If conflict numbers, last one will win\\
  \lstinline!\DIGIT! & back reference
\end{tabular}

\paragraph{Syntax code}
\texttt{\\sCODE} can be used to refer to any character "whose syntax is CODE".
\texttt{\\SCODE} is the negative form.
See the "Syntax Class Table" chapter of elisp manual for details.

\begin{tabular}{@{}ll@{}}
  \lstinline!\s-! & whitespace\\
  \lstinline!\s[space]! & whitespace\\
  \lstinline!\sw! & \lstinline!\w!\\
  \lstinline!\s.! & punctuation\\
\end{tabular}

\subsubsection{Search}
\begin{description}
\item[regexp-quote \NT{str}] \uline{a regexp} whose only exact match is str.
\item[regexp-opt \NT{strs}] \uline{an \textit{efficient} regexp}
  matching any of the strings.
\item[re-search-forward \NT{reg}] 
\item[re-search-backward \NT{reg}] 
\item[string-match \NT{reg str}] \uline{index} of first match.
  Set \uwave{\textit{Match Data}}
\item[string-match-p \NT{reg str}] \uline{index} of first match.
\item[match-string \NT{idx}] \uline{component} of \textit{Match Data}, 0 for entire match
\item[match-beginning \NT{idx}] \uline{beginning index} of the \textit{Match Data}
\item[match-end \NT{idx}]
\item[replace-regexp-in-string \NT{regexp rep string}] \uline{a
    modified copy} with all matches replaced with \texttt{rep}.
\end{description}

\subsection{Type}
\subsubsection{Equal}
\begin{description}
\item [eq \textnormal{\texttt{o1 o2}}] \texttt{t} if same Lisp object. Faster than \texttt{equal}
\item [equal \textnormal{\texttt{o1 o2}}] \texttt{t} if similar structure and contents
\item [=] \texttt{t} if all args are \texttt{equal}. Only for numbers
\item [string=] compare two strings
\end{description}

\subsubsection{Number}
\begin{description}
\item [expt \NT{a b}] $a^b$
\end{description}

\subsubsection{String}
\begin{description}
%  construct
\item [make-string \NT{ct char}] make string by duplicating characters
\item [substring \NT{str start \&opt end}]
\item [concat \NT{\&rest sequences}]
\item [split-string \NT{str \&opt seps}] default seps: \verb!"[ \f\t\n\r\v]+"!
% convert
\item [number-to-string \NT{num}]
\item [string-to-number \NT{str}]
\item [char-to-string \NT{char}]
\item [string-to-char \NT{str}] returns the first character in string
% case
\item [downcase \NT{string-or-char}]
\item [upcase \NT{string-or-char}]
\item [capitalize \NT{string-to-char}]
\end{description}

\subsubsection{cons cell}
This is a pair of slots, each of them can hold anything.  A list is a
list of cons cells, in which each cdr is the "address" next list.
\textit{Destructive} operations mean the cdr of the cons cells are
modified.

A \textit{dotted pair notation} is a more general syntax of a cons
cell.  It is written as \texttt{(A . B)}.  It is more general because
the cdr can also hold anything.  As an example, \texttt{(1 2)} is
the same as \verb$(1 . (2 . nil))$.  If the CDR of a list’s
last cons cell is some value other than ‘nil’, we call the structure a
\textit{dotted list}, since its printed representation would use
dotted pair notation.

\begin{description}
% construct
\item [cons \NT{obj1 obj2}] create a \uline{cons cell} with args
  as CAR and CDR. If obj1 is a list, this is called \textit{cons obj2 onto the list}.
\item [list \NT{\&rest objs}] create a list
\item [make-list \NT{length obj}] make a list of \texttt{length}, each
  cell holding the \textit{same} object, using \texttt{eq}
\item [append \NT{\&rest seqs}] last one is a list. Concatenate into
  one list, all arguments except the last one are copied. Set
  \texttt{nil} to the last to force copy everything.
\item [reverse \NT{list}] reverse the list by copying
\item [number-sequence \NT{from to}] the list of \texttt{[from,to]}, inclusive

% access
\item [car,cdr]
\item [car-safe,cdr-safe]
\item [caar] car car, 11
\item [cadr,cdar,cddr]
\item [nth \NT{n list}] get the \uline{nth element}, starting from 0
\item [nthcdr \NT{n list}] execute \uline{cdr n times}
\item [last \NT{list}]
\item [pop] return the \uline{cdr}, and \uwave{remove it from the list}
\item [length]

% modify  
\item [push \NT{el list}] modify \uwave{\texttt{list}} by cons \texttt{el} onto it.
\item [add-to-list \NT{list el}] cons \texttt{el} onto \uwave{\texttt{list}}
  if it is not there.
\item [setcar \NT{cons obj}]
\item [setcdr \NT{cons obj}]
\item [sort \NT{list pred}] destructive due to rearrange the cdrs
\end{description}

\subsubsection{set}
We use list as set, by ignoring the order ...
\texttt{append} to combine two set, then \texttt{delete-dups} to remove duplication ...

\begin{description}
\item [memq \NT{obj list}] whether obj is a member of list, using \texttt{eq}
\item [delq \NT{obj list}] destructively remove all elements \texttt{eq} to obj.
\item [rmq obj \NT{list}] return a copy of list with all obj removed
\item [member] \texttt{equal} counter-part
\item [member-ignore-case] for string
\item [delete] \texttt{equal} counter-part
\item [remove] \texttt{equal} counter-part
\item [delete-dups] use \texttt{equal}
\end{description}

\subsubsection{association list}
This is a special list, each element is a key-value pair.

\begin{lstlisting}
(setq alist-of-colors
  '((rose . red) (lily . white) (buttercup . yellow)))
\end{lstlisting}


\begin{description}
\item [assoc \NT{key list}]
  \uline{first element} of \textit{list} whose car \texttt{equal} \textit{key}.
\item [rassoc \NT{val list}]
  \uline{first element} of \textit{list} whose cdr \texttt{equal} \textit{val}
\item [assq \NT{KEY list}]
  \uline{first element} of \textit{list} whose car \texttt{eq} \textit{key}.
\item [rassq]
  \uline{first element} of \textit{list} whose cdr \texttt{eq} \textit{val}
\item [assoc-string \NT{key list \&opt case-fold}] This is \texttt{assoc} for
  string. if \texttt{case-fold} is non-nil, case is ignore-d.

\item [copy-alist \NT{list}] two-level deep copy
\item [assq-delete-all \NT{key list}] delete all elements whose car \texttt{eq} \textit{key}
\item [rassq-delete-all \NT{val list}] delete all elements whose cdr \texttt{eq} \textit{key}
\end{description}

\subsubsection{property list}
It is a flat list. The odd elements are property name, and the even elements are values.

\begin{lstlisting}[language=lisp]
(pine cones numbers (1 2 3) color "blue")
\end{lstlisting}
The property names \textit{must} be unique.
The order of the "pairs" does not matter.
\begin{description}
\item [plist-get \NT{plst prop}] get the \uline{value}
\item [plist-put \NT{plst prop val}] set the \uwave{value of \texttt{plst}}
\item [lax-plist-get] use \texttt{equal}
\item [lax-plist-put] use \texttt{equal}
\item [plist-member \NT{plst prop}] return \uline{the tail of
    \texttt{plst} whose car is \texttt{prop} (non-nil)} if \texttt{prop} is in
  \texttt{plst}. This is useful because it can distinguish the missing
  property and the property with value \texttt{nil}
\end{description}


\subsubsection{sequence}

\begin{tikzpicture}
  % \node (A) [rectangle,draw,rounded corners] at (0,1) {hello};
  % \node (B) at (0,2) {yes};
  % \draw (0,1) -- (2,4);
  % \node [rectangle,draw,label=sequence] {
  %   % \node [label=list] {
  %   %   % \node [label="Array"] {};
  %   % };
  % };

  \node (vector) at (0,0) [draw] {vector};
  \node (string) at (0,1) [draw] {string};
  \node (char) at (2,0) [draw] {Char table};
  \node (bool) at (2,1) [draw] {Bool vector};
  \node (array) [draw,fit={(vector) (string) (char) (bool)},"center:Array"] {};
  \node (list) at (-2,0) [draw] {List};
  \node (sequence) [draw,fit={(list) (array)},label={[shift={(-2,0)}]center:Sequence}] {};
\end{tikzpicture}

\begin{description}
\item [length \NT{seq}]
\item [elt \NT{seq idx}] returns the element of \textit{seq} at \textit{index}
\item [copy-sequence \NT{seq}] the sequence is new, but the elements are not.
\end{description}

\subsubsection{array}
It is fixed length sequence.
\begin{description}
% constructing
\item [make-vector \NT{length object}] create vector
\item [vector \NT{\&rest objects}] create vector
% accessing
\item [aref \NT{array index}] getter
\item [aset \NT{array index object}] setter
\end{description}

\subsubsection{hash table}
\begin{description}
% construct
\item [make-hash-table]

% access
\item [gethash \NT{key table}]
\item [puthash \NT{key value table}]
\item [remhash \NT{key table}] remove
\item [clrhash \NT{table}] remove all
\item [maphash \NT{function table}] call function once for each of the
  element in table. The function should accept two arguments: key and
  value
% other
\item [hash-table-count \NT{table}] return number of entries
\end{description}

\subsection{Function}

\begin{description}
  % define
\item [defun \NT{name arglst \&opt docstr decl \&rest body}]
\item [defmacro \NT{name args body}] Macro does not evaluate its
  arguments. It put the arguments /as is/ and put them into the macro
  body to form an expression.  The expression is then evaluated for
  result.

  % Anonymous functions
\item [lambda \NT{args body}]
\item [function \NT{fun-obj}] returns the function without evaluating
  it. It is the "quote" for function
\item [\#'] this is the read syntax for the above \texttt{function}
  special form. You see, it is indeed the \textit{quote} for the
  function.

  % Calling
\item [funcall \NT{fun \&rest args}] compute which function to execute
  at runtime
\item [apply \NT{fun \&rest args}] compute the arguments at
  runtime. Same as =funcall=, but the /last/ of arguments is a list,
  and will be expanded into many arguments instead of a list.

  % Mapping family
\item [mapcar \NT{fun seq}] execute function on each element of
  sequence, and return the list of results.
\item [mapc \NT{fun seq}] same as =mapcar= except it returns the
  =sequence=, with the intention to collect side effect.
\item [mapconcat \NT{fun seq sep}] execute function on
  elements of sequence. The results must be strings, and will be
  concatenated and returned.
\end{description}



\subsection{Control Structure}

\begin{description}
% Sequential
\item [progn \NT{forms...}] return the result of final form
\item [prog1 \NT{form1 forms...}] return the result of form1
\item [prog2 \NT{form1 form2 forms...}] return the result of form2
% Conditional
\item [not]
\item [and]
\item [or]
\item [if \NT{condition then-form else-forms...}]
\item [when \NT{condition then-forms...}]
\item [unless \NT{condition forms...}]
\item [cond \NT{clause...}] the clause must be a list:
  \texttt{(condition body-forms...)}.  It is not exactly the "case"
  statement, because the condition is evaluted to true or false.  Any
  remaining forms are \textit{ignored}.
\item [pcase \NT{exp branch1 branch2 branch3...}] this is more like
  the "case" statement. The EXP is first evaluted and compare with the
  car of each branches.  The branch must be of the form
  \texttt{(UPATTERN BODY-FORMS...)}
\item [while \NT{condition forms...}]
\item [dolist \NT{(var list [result]) body...}] execute body for each
  element of list, with the bound of var to the current element and
  result for return.
\item [dotimes \NT{(var count [result]) body...}] execute body for
  each index of \texttt{[0,count)}, with var bound to the index, and
  result bound for return.
\end{description}

\subsubsection{CL Loop facility}
The package cl-lib.el ports many common lisp facilities into elisp,
most importantly, the loop facility.  So this section, at least for
now, focus on \texttt{cl-loop}.

\begin{description}
\item [cl-loop \NT{clauses...}] general loop form.
\item [for \NT{var} from \NT{from} to \NT{to} by \NT{step}]
  \textit{from} defaults to 0. \textit{step} must be positive and default to 1;
  inclusive \texttt{[from,to]};
  \texttt{from} can be \texttt{upfrom} and \texttt{downfrom}. I think it is wired to use this;
  \texttt{to} can be \texttt{upto} and \texttt{downto}. This makes more sense;
  \texttt{above} and \texttt{below} can be used, but /exclusive/. e.g. \texttt{for var below 10}
\item [for VAR in LIST by FUNCTION] =FUNCTION= is used to traverse the list, defaults to =cdr=
\item [for VAR on LIST by FUNCTION] =VAR= is bound to the cons cell of the list instead of the element.
\item [for VAR across ARRAY] iterates all elements of array
\item [for VAR ]PR1 then EXPR2= :: this is the most general form.  The
  =VAR= is bound to =EXPR1= initially, and will be set by evaluating
  =EXPR2= in successive iterations.  =EXPR2= can refer the old =VAR=

% iteration clauses
\item [repeat \NT{integer}] repeat the loop how many times
\item [while \NT{condition}] stops the loop when the condition becomes nil
\item [until \NT{condition}]
\item [always \NT{condition}] like while except it returns =nil=, and =finally= clauses are not executed.
\item [never \NT{condition}] counter part for =always=

% accumulation clauses
\item [collect \NT{form}] collect into a list and return the list in the end
\item [append \NT{form}] collect the lists into a list by appending, and return it in the end
\item [concat \NT{form}] for string only
\item [count \NT{form}] count how many times form evaluates to non-nil.
\item [sum \NT{form}] sum all the values
\item [maximize \NT{form}] get the max. If the form is never executed, result is /undefined/
\item [minimize \NT{form}]

% Other clauses
\item [with var = value] set the value one-time at the beginning of
  the loop.  Often used as return variable.  *The spaces around ~=~ is
  essential!*.
\item [if condition clause [else clause]]
\item [when condition clause] same as if
\item [unless condition clause] similar
\item [initially [do] forms...] execute before the loop begins, but
  after the =for= and =with= variable bindings. =do= is optional.
\item [finally [do] forms...] execute after the loop finishes
\item [finally return form] finally return it ...
\item [do forms...] execute as an implicit =progn= in the body
\item [return form] this is often used in =if= or =unless=, because
put it in top level will cause the loop always execute only once.
\end{description}



\subsection{Debugging}
\subsubsection{lisp debugger}
The simplest debugger is called =lisp debugger=.
You can turn on the =debug-or-error= flag,
but I found inserting the =(debug)= command useful.
Simply insert =(debug)= where you want program to suspend, and run it.
You will enter the debugger at that point.
In the debugger buffer, the following commands are available:
\begin{description}
\item [c] continue run program
\item [d] step
\item [e] evaluate an prompt expression
\item [R] like =e=, but also save the result in =*Debugger-record*=
\item [q] quit
\item [v] toggle display of local variables ???
\end{description}
\subsubsection{Edebug}
For this to work, first you need to instrument the code.
You can instrument the defun by =C-u C-M-x=.
Actually this is adding a prefix before =eval-defun=,
which instrument, and then evaluate the defun.

After instrumentation, running the defun will cause the program to stop at the first /stop point/ of the function.
The /stop points/ are
\begin{itemize}
\item before and after each subexpression that is a list
\item after each variable reference
\end{itemize}

\begin{description}
\item [b] set a breakpoint. You can also set the /source breakpoints/, by adding =(edebug)=.
\item [u] unset a breakpoint
\item [x CONDITION] set a conditional breakpoint
\item [B] move point to the next breakpoint
\item [w] move point back to the current stop point
\item [<SPC>] run to next stop point
\item [g] execute until next breakpoint
\item [q] exit
\item [S] stop and wait for Edebug commands
\item [n] evaluate a sexp and stop at stop point
\item [t] /trace/, pause one second at each stop point ...
\item [T] rapid trace. Update the display at each stop point but don't actually pause ...
\item [c] pause one second at each breakpoint
\item [C] rapid continue.
\item [G] run and ignore breakpoints (but you can stop it by =S=)

\item [h] proceed to the stop point near the point ...
\item [f] run one expression
\item [o] step out the containing expression
\item [i] step in
\item [e EXP] evaluate a prompt expression
\item [C-x C-e] evaluate an expression at point
\item [?] show help
\item [r] redisplay the most recent sexp result
\item [d] display the backtrace
\end{description}


%%% Local Variables:
%%% TeX-master: "cheatsheet"
%%% End:
