\section{sed}

\subsection{Introduction}
The sed version on Mac OS and GNU Linux are different.
So, use gnu! On Mac, install
\begin{lstlisting}
brew install gnu-sed
\end{lstlisting}

This will make a =gsed= command available.
To write a cross platform script, use
\begin{lstlisting}
echo "OSTYPE: " $OSTYPE
SED=sed
if [[ "$OSTYPE" == "linux-gnu" ]]; then
    SED=sed
elif [[ "$OSTYPE" == "darwin"* ]]; then
    SED=gsed
fi
$SED -E -e "460,$ s/REG[0-9]{1,2}//g" compress42.c.orig > compress42.bugsig.c
\end{lstlisting}

\subsubsection{About the regular expression version}
\texttt{-E} will enable extra features, such as: \texttt{a{1,2}}
See \verb$re_format(7)$ for details.

There's no =\d=, so use =[0-9]= instead. The man page says =[:digit:]= can be used, but it seems not working.

\subsection{usage}
in shell script

\begin{lstlisting}
#!/bin/sed -f
#!/bin/sed -nf
\end{lstlisting}

\subsection{concepts}
\begin{description}
\item [input]
\item [output]
\item [pattern space]
\item [hold space] This is a spare pattern space, used to remember the
  data in pattern buffer
\end{description}

\subsection{workflow}
\begin{enumerate}
\item copy a line from input(exclude the tailing newline) into pattern buffer
\item apply command(s) to it
\item output
\end{enumerate}

\subsection{cmd args}
\begin{description}
\item [-n] by default each line of input is echoed to the standard
  output after all of the commands have been applied to it. The -n
  option suppresses this behavior
\item [-e] sed -e 'xxx' -e 'xxx' -e 'xxx'
\item [-f] script file
\end{description}


\subsection{range}
sed will only apply command for the lines in the specified range.
If the command is preceded with =!=, that means the command works on lines not in the range.

\begin{lstlisting}
sed '1,100 s/A/a/' # by line number
sed '101,$ s/A/a/' # $ is last line
sed '/start/,/stop/ s/#.*//' # by pattern
\end{lstlisting}

\subsection{commands}

edit
\begin{tabular}{l|l|l}
shortcut & name & description\\
\hline
a & add & \\
i & insert & \\
c & change & \\
d & delete & \\
s & substitute & \\
 &  & \\
\end{tabular}

output

\begin{tabular}{l|l|l}
shortcut & name & description\\
\hline
= & print & \\
l & look & \\
p & print & \\
n & next & \\
r & read & \\
w & write & \\
\end{tabular}

flow control

\begin{tabular}{l|l|l}
shortcut & name & description\\
\hline
q & quit & \\
b & branch & \\
t & test & \\
:label &  & \\
\end{tabular}
\subsection{examples}

print

\begin{lstlisting}
# add line numbers first,
# then use grep,
# then just print the number
cat -n file | grep 'PATTERN' | awk '{print $1}'
# the equilvalence
sed -n '/PATTERN/ =' file
\end{lstlisting}

substitute

\begin{lstlisting}
s/pattern/&/ # '&' stands for the total match
# in extend mode(-E), can use \1 \2
s/(a)b/\1/
s//string/ # use the last run-time used pattern
s/xxx/xxx/g # substitute globally: all
# there will not be recursion. sed will not examine the generated string
s/loop/loop loop/g # will NOT run forever
s/xxx/xxx/2 # only substitute the second match
s/xxx/xxx/g2 # substitute 2,3,4,...
s/xxx/xxx/p # will print out even if -n is used
s/xxx/xxx/I p # ignore case; command can be used together
s/a/A/2pw /tmp/file # combine more
\end{lstlisting}

delete

\begin{lstlisting}
# -i: make change to the original file
# /d: delete the line if match
sed -i '/@slice/d' $ClassName.java
sed -i 'g/@slice/d' xx.java # remove all
sed '/^$/d' # remove all empty lines
sed '11,$ d' # only output first 10 lines
sed '1,/^$/ d' # delete everything up to the first blank line.
\end{lstlisting}


%%% Local Variables:
%%% TeX-master: "cheatsheet"
%%% End:
