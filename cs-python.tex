\section{Python}
\subsection{Scoping}
There're four levels:
\begin{itemize}
\item current scope
\item parent scope
\item module scope (global)
\item built-in scope
\end{itemize}

\texttt{nonlocal} keyword specify this variable should be referenced
to the parent scope.  But, this will not reach global.  Instead, the
=global= keyword declares the listed variables to be in the module
level scope.

\begin{quote}
The nonlocal statement causes the listed identifiers to refer to previously bound variables in the nearest enclosing scope excluding globals.
\end{quote}

As an example:
\begin{lstlisting}[language=python]
var = 0 # global

def outer():
  var = 1 # parent
  def inner():
    nonlocal var
    var = 2 # local
    global var
    var =3
  inner()
  # var = 2

outer()
# global var = 3
\end{lstlisting}



\subsection{Collection}

\subsubsection{String}

\paragraph{Concatenation}
\begin{itemize}
\item concatenate two strings directly by =+=.
\item need to convert integer to string before concatenate: =s + str(35)=
\end{itemize}

\paragraph{split}
\begin{description}
\item [str.split(sep=None)] default by white space
\item [str.strip()] strip out white space at both begin and end
\item [str.replace(old, new)] replace /all/.
\item [str.startswith(s)]
\item [str.endswith(s)]
\end{description}

\subsubsection{tuple}
TODO
\subsubsection{ List}
\paragraph{Slicing}
The slicing syntax is \texttt{l[start:end:step]}.
The slicing will return a \textit{new} list. Change to that list will not change the original one.

\begin{lstlisting}
l[4]
l[4:]
l[::2]
l[:-1]
\end{lstlisting}

However, assign to the slicing itself /will change/ the original one:
\begin{lstlisting}
l[1:2] = [4,5,6]
\end{lstlisting}

Also, assign to a new variable only assign the reference:
\begin{lstlisting}
a = [1,2,3]
b = a # only a reference
\end{lstlisting}

\paragraph{create a list}
\begin{description}
\item [range(stop)]
\item [range(start, stop[, step])]
\end{description}


\subsubsection{Dictionary}
Create:
\begin{lstlisting}
x = {'a': 1, 'b': 2}
\end{lstlisting}

Dictionary is not sorted. Use \texttt{collections.OrderedDict} if you
want this feature.  Basically it remember the order when the elements
are inserted.

\begin{lstlisting}
import collections
od = collections.OrderedDict(sorted(d.items()))
\end{lstlisting}

Merge two dictionary (=x= and =y=):
\begin{lstlisting}
z = x.copy()
z.update(y)
\end{lstlisting}


\subsubsection{Set}
\begin{lstlisting}
s = set()
s.add(x)
if x in s:
  pass
\end{lstlisting}




\subsection{Algorithm}
\subsubsection{sort}
TODO

sort a dictionary by value:
\begin{lstlisting}
sorted(dict1, key=dict1.get) # => list
sorted(dict1, key=dict1.get, reverse=True)
\end{lstlisting}


\subsection{Function}
\subsubsection{variadic parameter}
use =*args= syntax, and =args= will be a /tuple/:
\begin{lstlisting}
  def foo(*args):
    for a in args:
      print a
\end{lstlisting}

use =**args= to capture all /keyword arguments/.

\begin{lstlisting}
def bar(**kwargs):
  for a in kwargs:
    print a, kwargs[a]
\end{lstlisting}

Combine them together:
\begin{lstlisting}
def foobar(kind, *args, **kwargs):
  pass
\end{lstlisting}

Also, there's a concept for the reverse thing: unpack argument list from a list, with =*list=:
\begin{lstlisting}
def foo(a,b):
  pass

l = [1,2]
foo(*l)
\end{lstlisting}

on python3, this syntax can appear on left side
\begin{lstlisting}
first, *rest = [1,2,3,4]
first,*l,last = [1,2,3,4]
\end{lstlisting}

\subsection{Exception}
To give a quick feel:
\begin{lstlisting}
try:
  pass
except TypeError as e: # capture the exception into a variable
  pass
except AnotherError: # does not capture
  pass
except: # all exception
  pass
else: # if doesn't raise an exception
  pass
finally:
  pass
\end{lstlisting}

\subsection{Lambda}
\begin{lstlisting}
lambda x : x+2
lambda x: x%2==0
\end{lstlisting}

The usage of lambda is often in /map/ and /filter/.
% \begin{description}
% \item [map(lambda_exp, mylist)] will execute the lambda expression on
%   each element of the list, and return a list containing the results.
% \end{description}



%%% Local Variables:
%%% TeX-master: "cheatsheet"
%%% End:
