\section{SQL}

SQL is VERY sensitive with the terminating colon!

\subsection{Emacs Babels}
There're two babels: \texttt{sqlite} and \texttt{sql}. The SQL babel
works for a RDBMS like MySQL.  But a file-based sqlite database is
easier to use, so we talk about \texttt{sqlite} babel here.

To create a table, you probably want it to be silent.  Use this header
\verb$#+header: :results silent$ to do that.

You need a database file of course.  \verb$#+header: :dir ~/tmp/$ and
\verb$#+header: :db test-sqlite.db$ together will locate the database
file.

To show the result, the default is in table format, like most babels
for emacs.  If you also want the column names like those shown in
command line interface, add \verb$#+header: :colnames yes$

In stead of showing as a table by default,
\verb$#+header: :results raw$ will print the raw text.  However the
result will be appended if running multiple times instead of
replacing.

The following is an example.

\begin{lstlisting}
#+name: sqlite-populate-test
#+header: :results silent
#+header: :dir ~/tmp/
#+header: :db test-sqlite.db
#+BEGIN_SRC sqlite
create table greeting(one varchar(10), two varchar(10));
insert into greeting values('Hello', 'world!');
#+END_SRC
\end{lstlisting}

\begin{lstlisting}
#+name: sqlite-populate-test
#+header: :colnames yes
#+header: :dir ~/tmp/
#+header: :db test-sqlite.db
#+BEGIN_SRC sqlite
select * from greeting;
#+END_SRC
\end{lstlisting}

\subsection{Basic Statements}
create, insert and delete
\begin{lstlisting}
  create table greeting(one varchar(10), two varchar(10));
  insert into greeting values('Hello', 'world!');
  insert into tablename values (1, 'xxx', 34);
  insert into tablename (col1, col3) values (1, 34);
  update tablename set col1=3,col3=8 where col2='yyy';
  delete from tablename where col1=3;
  delete from tablename; -- same as:
  delete * from tablename;
\end{lstlisting}

\subsection{select}
\begin{lstlisting}
  select * from tablename;
  select col1,col2 from tablename;
  select distinct col1,col2 from tablename;
  -- 'xxx' depends on the type of that column, e.g. for integer, 123
  -- operators can be =, <=, <>(not equal)
  select * from tablename where col1='xxx';
  select * from tablename where col1='xxx' and col2<>3 or col3=8;
  select * from tablename order by col1 ASC, col2 DESC, col3;
  select * from tablename where col1 in ('xxx', 'yyy');

  select top 2 * from tablename;
  select top 50 percent * from tablename;
  select * from tablename limit 3;
\end{lstlisting}

\subsection{Aliasing}
\begin{lstlisting}
SELECT o.OrderID, o.OrderDate, c.CustomerName
FROM Customers AS c, Orders AS o
WHERE c.CustomerName="Around the Horn" AND c.CustomerID=o.CustomerID;
\end{lstlisting}

\subsection{SQLite}
This is a file-based database.  There's no server, every read and
write goes into the ordinary file.

There're some special commands for sqlite command line client, the dot commands.
To see the help, use =.help=.
Some useful ones:
\begin{description}
\item [.tables ?TABLE?] show the tables (optionally only the tables that match the pattern =?TABLE?=).
  There's no statements like =show tables;= in MySQL.
\item [.schema ?TABLE?] similar to =describe tablename;=
\end{description}


\subsubsection{Python Interface}
\begin{lstlisting}
import sqlite3
conn = sqlite3.connect('example.db')
c = conn.cursor()
c.execute('''CREATE TABLE ...''')
conn.commit()
conn.close()
\end{lstlisting}

\begin{lstlisting}
import sqlite3
conn = sqlite3.connect('example.db')
c = conn.cursor()
t = ('RHAT',)
c.execute('SELECT * FROM stocks WHERE symbol=?', t)
print(c.fetchone())
\end{lstlisting}



%%% Local Variables:
%%% TeX-master: "cheatsheet"
%%% End:
