\section{Prolog}

online resources
\begin{itemize}
\item learn prolog now \footnote{http://lpn.swi-prolog.org/lpnpage.php?pageid=top}
\item swipl library \footnote{http://www.swi-prolog.org/pldoc/man?section=lists}
\end{itemize}

\subsection{Running program}

Install swi-prolog, and emacs should be able to support it.  In a
prolog file, select a region, and \texttt{C-c C-c r} (execute region).
Using this for facts and rules, but evalute query direct in the repl.

\subsection{From the slides}

A Horn Clause: $c \quad h_1 \wedge h_2 \wedge \ldots \wedge h_n$.
$c$ is the consequent, the conjunction of $h_i$ is the antecedent.
It means if all $h_i$ are true, $c$ is true.
This is written in prolog as:

\begin{lstlisting}
c :- h1,h2,...,hn
\end{lstlisting}

Logic program is actually a collection of Horn Clauses.
A logic problem has three components:
\begin{description}
\item [fact] Horn Clause with no Antecedent. isamother(mary)
\item [rule] Horn Clause with Antecendent
\item [query] Horn Clause with no Consequent
\end{description}

Example:

\begin{lstlisting}[language=prolog]
  % facts
  isamother(mary).
  childof(tom, mary).
  % rules
  loves(mary, tom) :- isamother(mary), childof(tom, mary).
  % quries
  ?- loves(mary, tom).
\end{lstlisting}

Prolog can have variables. It is often used in rules and quries.
\begin{lstlisting}
  loves(X, Y) :- isamother(X), childof(Y, X).
  ?- loves(mary, X).
\end{lstlisting}

Syntax of prolog:
\begin{lstlisting}
Clause ->  Predicate.
         | Predicate :- PredicateSeq.
PredicateSeq ->  Predicate
               | Predicate, PredicateSeq
Predicate -> PredName(TermSeq)
TermSeq   ->  Term
            | Term, TermSeq
Term      ->  FunctorName(TermSeq) | Constant | Variable
\end{lstlisting}

\subsection{Executing}
\subsubsection{Unification}
Given two atomic formula (predicates), they can be unified if and only
if they can be made syntactically identical by replacing the variables
in them by some terms.

Unify childof(jane, X) and childof(jane, mary).

\textit{MGU} results from a substitution that bounds free variables as little
as possible

\subsubsection{Substitution}
Substitution maps variables to terms.  Instantiation is the
application of substitution to all variables in a prolog formula,
term.

Unify $childof(jane, X)$ and $childof(Y, mary)$ by $[X \rightarrow mary, Y \rightarrow jane]$

\subsubsection{Unification and Computing with Logic}

Given a query
\begin{itemize}
\item Search the facts and rules to find whether the query unifies
  with any consequent
\item If the search fails, return false (query result)
\item If the search is successful, then
  \begin{itemize}
  \item if the unification occurs with the consequent of a fact,
    return the substitution of the variables (if any)
  \item if the unification occurs with the consequent of a fact,
    return the substitution of the variables (if any)
  \end{itemize}
\end{itemize}

An example:
\begin{lstlisting}
isamother(mary).
childof(tom, mary).
loves(X, Y) :- isamother(X), childof(Y, X).
?- loves(mary, tom).
\end{lstlisting}

\subsubsection{Backtracking}

\begin{itemize}
\item Re-try to unify with some clause other than the fact, and proceed
\item Re-try to unify with some clause other than the fact, and proceed
\end{itemize}

Example:
\begin{lstlisting}
isamother(mary).
childof(jane, mary).
childof(tom, mary).
loves(X, Y) :- isamother(X), childof(Y, X).
?- loves(mary, X)
\end{lstlisting}

An graph example:
\begin{lstlisting}
edge(a, b).
edge(b, c).
edge(c, a).
reach(X, Y) :- edge(X, Y).
reach(X, Y) :- edge(X, Z), reach(Z, Y).
?- reach(a, c)
\end{lstlisting}

\subsection{Other Prolog Features}
\subsubsection{List}

\begin{description}
\item [[a, b, c]] a list of 3 elements
\item [[]] empty list
\item [[a, [b, c], [[d, e]], []]] list can contain different type
\item [[a|[b, c]]] head and tail
\end{description}

Example:
\begin{lstlisting}
  ?- [1, 2, 3] = [X|Xs].
  % X=1
  % Xs = [2, 3]
  ?- [1, 2, 3] = [X|[Y|Rest]].
  % X=1
  % Y=2
  % Rest = [3]
\end{lstlisting}

\subsubsection{if-then-else}
\begin{lstlisting}
  ?- Z = 3, (Z == 3 -> X = 1, Y = 2; X = 2, Y = 1).
  % Z=3
  % X=1
  % Y=2
\end{lstlisting}

\subsection{Examples}
\begin{lstlisting}
lectures(monday, nolecture).
lectures(tuesday, vp).
lectures(tuesday, se).
lectures(tuesday, ddbms).
lectures(wednessday, ds).
lectures(wednessday, mpl).
lectures(thursday, vp).
lectures(thrusday, se).
lectures(friday, ds).
lectures(friday, mpl).
lectures(saturday, ai).
lectures(saturday, ddbms).
%% ?- lectures(friday, X), write(X),nl.
%% ?- lectures(friday, X), write(X), nl, fail.
\end{lstlisting}



%%% Local Variables:
%%% TeX-master: "cheatsheet"
%%% End:
